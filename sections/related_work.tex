Bereits vor der Entwicklung von Memoryland, gab es bestehende Lösungen zur Verwaltung
und Präsentation von Erinnerungen. Bevor mit dem Projekt begonnen wurde, wurden
diese analysiert und die Erkenntnisse zu den Stärken und Schwächen dieser Systeme
notiert. Daraus entstanden dann die Anforderungen für das finale System.

\section{Analyse der vorhandenen Systeme}

\subsection{OneDrive}

\emph{Vorteile:}
Das System bietet eine cloudbasierte Speicherung für Bilder und andere Medien. 
So können Nutzer ihre Dateien zentral ablegen und von verschiedenen Geräten aus darauf 
zugreifen.

\emph{Nachteile:}
Eine Integration von Bildern in interaktive Formate ist nicht vorhanden. 
Dadurch bleibt die Nutzung der gespeicherten Bilder auf eine einfache 
Anzeige und Verwaltung beschränkt.

\emph{Zusammenfassung:}
OneDrive bietet eine sichere und cloudbasierte Speicherung von Bildern und anderen 
Medien. Es fehlen Funktionen zur Bildbearbeitung und zur Integration von 
immersiven Formaten.

\footnote{Informationen zu OneDrive stammen von \cite{MicrosoftCorporation}}

\subsection{Google Photos}

\emph{Vorteile:}
Das System stellt eine Oberfläche für eine Verwaltung und Anzeige von Bildern 
zur Verfügung. Dadurch können Nutzer ihre Bilder organisieren und betrachten.

\emph{Nachteile:}
Eine Integration von Bildern in interaktive Formate ist nicht vorhanden. 
Somit ist es nicht möglich sie in erweiterte Präsentationsformen einzubinden 
und beschränkt die Nutzung der Bilder auf Verwaltungs- und Anzeigezwecke.

\emph{Zusammenfassung:}
Google Photos ermöglicht eine Verwaltung und Anzeige von Bildern, bietet jedoch 
keine Möglichkeit zur Umwandlung von Bildern in interaktive Formate.

\footnote{Informationen zu Google Photos stammen von \cite{GoogleIrelandLimited}}

\subsection{Animoto}

\emph{Vorteile:}
Das System ermöglicht die Umwandlung von Bildern in animierte Diashows,
wodurch Bilder in eine dynamische Präsentationsform dargestellt werden
können.

\emph{Nachteile:}
Die erstellten Diashows enthalten jedoch keine immersiven Komponenten.

\emph{Zusammenfassung:}
Animoto bietet die Möglichkeit, Bilder in Diashows umzuwandeln. Diese Funktion gleicht 
einem animierten Fotoalbum und bietet keine immersiven Erlebnisse, wie sie in Memoryland
vorgesehen sind.

\footnote{Informationen zu Animoto stammen von \cite{Animoto}}

\subsection{Zusammenfassung}

Die Marktanalyse zeigt, dass zwar verschiedene Plattformen grundlegende 
Funktionen zur Speicherung und Präsentation von Bildern bieten, jedoch 
keine immersiven Erlebnisse ermöglichen. OneDrive und Google Photos kümmern 
sich um die sichere Speicherung und Verwaltung. Animoto ermöglicht die 
Erstellung von Diashows, jedoch ohne die Möglichkeit, Bilder in virtuelle 
Umgebungen zu integrieren.

\section{Funktionale Anforderungen}

\subsection{Benutzerverwaltung}

Das System soll eine Benutzerverwaltung bereitstellen, die eine Registrierung 
und Authentifizierung ermöglicht. Für eine sichere Authentifizierung erfolgt
die Anmeldung über Azure AD B2C.

\subsection{Bilder-Upload}

Nutzer sollen die Möglichkeit haben, Bilder hochzuladen und in Alben zu organisieren. 
Um Erinnerungen gut organisieren zu können sollen hochgeladene Bilder und Alben 
umbenannt werden können. Darüber hinaus soll das System eine Suchfunktion bieten, 
damit Nutzer ihre Bilder schnell wiederfinden können. Um den Upload-Prozess zu 
vereinfachen, soll es möglich sein, mehrere Bilder auf einmal hochzuladen. Falls 
große Mengen an Bildern übertragen werden, soll ein Resumable Upload verwendet werden
können, sodass unterbrochene Uploads fortgesetzt werden können.

\subsection{Präsentation der Erinnerungen}

Das System soll es ermöglichen, mehrere Memorylands zu erstellen, die dem 
gleichen oder unterschiedlichen Typen angehören können. Ein Typ definiert 
dabei eine eigene Szene, z.\,B.\ eine Wald- oder Inselumgebung. Nutzer sollen
in einem Memoryland ihre Bilder an bestimmten Stellen platzieren können, um 
eine immersive Darstellung ihrer Erinnerungen zu ermöglichen.

\subsection{Sharing-Funktion}

Memorylands sollen mit anderen Personen geteilt werden können. Dies soll es 
Nutzern ermöglichen, ihre Erinnerungen mit Familie und Freunden zu teilen.

\subsection{Security}

Die hochgeladenen Bilder und Alben sollen ausschließlich für den jeweiligen Nutzer verfügbar
sein. Andere Nutzer sollen keinen Zugriff auf fremde Bilder erhalten. Wenn Memorylands mit 
anderen geteilt werden, soll deshalb eine deutliche Warnung darauf hinweisen, dass die Inhalte
für andere sichtbar werden. Zudem soll es jederzeit möglich sein, eine Freigabe wieder 
zurückzuziehen, sodass geteilte Memorylands ungültig werden und somit nicht mehr 
aufrufbar sind.

\section{Nicht funktionale Anforderungen}


\subsection{Security}

Die Benutzeranmeldung erfolgt ausschließlich über Azure AD B2C, um eine sichere 
Authentifizierung zu gewährleisten. Hochgeladene Bilder und Memorylands dürfen nur 
für den jeweiligen Nutzer zugänglich sein. Die Datenübertragung erfolgt über 
TLS-verschlüsselte Verbindungen (HTTPS). Zudem Nutzer sollen jederzeit die Möglichkeit
haben, ihre Daten zu löschen („Recht auf Vergessenwerden“).

\subsection{Benutzerfreundlichkeit \& Design}

Die Webanwendung muss eine intuitive Benutzeroberfläche bieten, sodass Nutzer ihre 
Erinnerungen möglichst einfach hochladen, verwalten und präsentieren können. 
Suchleisten und das Sortieren der Daten erleichtert den schnellen Zugriff auf 
unterschiedliche Inhalte.
