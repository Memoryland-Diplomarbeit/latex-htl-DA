\subsection{Redux-Architektur und seine drei Prinzipien}

Die steigende Komplexität von SPAs (Single-Page-Anwendungen) führt zu einer zunehmenden Herausforderung 
beim Zustandsmanagement. Ungeregelte Zustandsänderungen können dazu führen, dass zum Beispiel ein Teil
des Programmes die Änderungen mitbekommt, ein anderer Teil jedoch nicht. Redux adressiert diese Problematiken 
durch drei grundlegende Prinzipien:

\textbf{Single Source of Truth:} Der gesamte Anwendungszustand wird in einem einzigen, zentralen ``Store'' als 
Objekt gespeichert. Dies erleichtert die Serialisierung, Wiederherstellung und Übertragung des Zustands zwischen 
Server und Client. Zudem ermöglicht es eine vereinfachte Fehlersuche und Debugging. Durch die zentrale Speicherung 
werden auch komplexe Funktionen wie Undo/Redo leichter implementierbar. (siehe Listing \ref{lst:single-source-of-truth})

\begin{lstlisting}[numbers=left,caption={Single Source of Truth Implementierung in Memoryland},label={lst:single-source-of-truth}]
    export interface Model {
        ...
        memorylands: Memoryland[];
        searchConfigList: string;
        uploadAlbumModel: UploadAlbumModel;
        transaction: Transaction | undefined;
        loadingAlbums: boolean;
        ...
    }
    ...
    export const store = new BehaviorSubject<Model>(initialState);
\end{lstlisting}

\textbf{State ist unveränderlich:} Änderungen am Zustand erfolgen ausschließlich über Aktionen. Eine Aktion 
ist ein Objekt, das beschreibt, welche Änderung stattfinden soll. Dadurch wird sichergestellt, dass weder die 
Benutzeroberfläche noch Netzwerk-Callbacks direkt auf den Zustand zugreifen. Dies minimiert potenzielle Race 
Conditions. Da Aktionen einfache Objekte sind, lassen sie sich protokollieren, speichern und für Debugging-Zwecke 
erneut abspielen. (siehe Listing \ref{lst:state-management})

\textbf{Zustandsänderungen durch reine Funktionen:} Änderungen am Zustand erfolgen durch sogenannte Reducer. 
Diese sind reine Funktionen, die den vorherigen Zustand und eine Aktion als Eingabe erhalten und daraus einen 
neuen Zustand ableiten. Der vorherige Zustand wird nicht direkt verändert. Reducer können modular aufgebaut 
werden, sodass einzelne Teile des Zustands unabhängig verwaltet werden. (siehe Listing \ref{lst:state-management})

\begin{lstlisting}[numbers=left,caption={State ist unveränderlich Implementierung in Memoryland},label={lst:state-management}]
    export function set(recipe: (model: Draft<Model>) => void) {
        const nextState = produce(store.value, recipe);
        store.next(nextState);
    }

    set(model => {
        model.selectedMemorylandType = id;
    });
\end{lstlisting}

Durch diese Prinzipien wird eine nachvollziehbare Zustandsverwaltung ermöglicht.
\footnote{Alle Informationen zu der Redux-Architektur und seinen drei Prinzipien stammen von \cite{DAR} \& \cite{DARa}}

\subsection{Image Previews}
Die Methode ``createImagePreview'' (siehe Listing \ref{lst:method-create-image-preview}) ermöglicht die Erzeugung 
einer Image-Preview mit einem festen Seitenverhältnis von 16:9. Dabei wird das ursprüngliche Bild in ein 
``canvas''-Element gezeichnet und auf eine feste Größe skaliert. Die Methode gibt eine ``Promise<string>'' zurück, 
die nach erfolgreicher Verarbeitung eine Data-URL des generierten Bildes liefert.

Data-URLs ermöglichen das Einbetten kleiner Dateien direkt in Dokumente mit dem Präfix data:. Sie bestehen 
aus einem MIME-Typ, optional einem base64-Token für binäre Daten und den eigentlichen Daten. Diese URLs sind 
besonders nützlich für das Einfügen von Text, Bildern oder HTML-Inhalten in Webseiten. Bei uns dienen sie dazu,
dass wir die Previews nicht ständig neu generieren müssen. 
\footnote{Alle Informationen zu Data-URLs stammen von \cite{MozillaFoundationa}}

Zunächst wird ein ``Image''-Objekt erstellt, das die Bildquelle (``imageUrl'') lädt. Nach dem Laden des Bildes wird ein 
``canvas''-Element erzeugt, dessen Kontext (``ctx'') für das Zeichnen erforderlich ist. Falls dieser nicht verfügbar 
ist, wird die Promise mit einer Fehlermeldung abgelehnt.

Das Bild wird basierend auf dem Zielseitenverhältnis (16:9) zugeschnitten. Falls das ursprüngliche Bild ein anderes 
Seitenverhältnis hat, werden entweder Breite oder Höhe entsprechend angepasst, sodass das Bild möglichst viel vom 
ursprünglichen Inhalt beibehält. Dies erfolgt durch Berechnung eines geeigneten Ausschnitts (``cropWidth'' und ``cropHeight''), 
sowie eines Offsets (``offsetX'', ``offsetY''), um das Bild mittig zu beschneiden.

Anschließend wird das Bild in den ``canvas'' gezeichnet und die erzeugte Vorschau in das gewünschte Format 
(``image/png'' oder ``image/jpeg'') konvertiert. Das Ergebnis wird als Data-URL zurückgegeben. Falls ein Fehler 
auftritt, etwa beim Laden des Bildes, wird die Promise entsprechend abgelehnt.

Diese Methode ermöglicht eine einheitliche Bildvorschau mit festem Seitenverhältnis, was es Memoryland-Benutzer:innen
leichter macht, das Programm zu bedienen und ihre Bilder zu erkennen.

\begin{lstlisting}[numbers=left,caption={Methode createImagePreview},label={lst:method-create-image-preview}]
    private readonly height: number = 9*6;
    private readonly width: number = 16*6;
  
    createImagePreview(imageUrl: string, isJpg: boolean): Promise<string> {
      return new Promise((resolve, reject) => {
        const img = new Image();
        img.crossOrigin = 'Anonymous';
        img.src = imageUrl;
        img.onload = () => {
          const canvas = document
            .createElement('canvas');
  
          const ctx = canvas
            .getContext('2d');
  
          if (!ctx) {
            reject('Error getting canvas context');
            return;
          }
  
          canvas.width = this.width;
          canvas.height = this.height;
  
          const targetAspectRatio = 16 / 9;
          const imgAspectRatio = img.width / img.height;
  
          let cropWidth = img.width;
          let cropHeight = img.height;
          let offsetX = 0;
          let offsetY = 0;
  
          if (imgAspectRatio > targetAspectRatio) {
            cropHeight = img.height;
            cropWidth = img.height * targetAspectRatio;
            offsetX = (img.width - cropWidth) / 2;
          } else {
            cropWidth = img.width;
            cropHeight = img.width / targetAspectRatio;
            offsetY = (img.height - cropHeight) / 2;
          }
  
          ctx.drawImage(
            img,
            offsetX,
            offsetY,
            cropWidth,
            cropHeight,
            0,
            0,
            this.width,
            this.height
          );
  
          let contentType: string = 'image/png';
  
          if (isJpg) {
            contentType = 'image/jpeg';
          }
  
          const previewDataUrl = canvas
            .toDataURL(contentType);
  
          resolve(previewDataUrl);
        };
        img.onerror = reject;
      });
    }
\end{lstlisting}

\subsection{Integration der Unity WebGL}
\label{sec:frontend-integration-webgl}

Im Projekt wurden die kompilierten Unity WebGL-Dateien im public-Ordner der Angular-Anwendung abgelegt. Der 
public-Ordner dient dazu, statische Dateien wie Bilder, CSS-Dateien oder JavaScript-Dateien zu speichern, 
die vom Entwicklungsserver als solche bereitgestellt und unverändert in das Build-Verzeichnis übernommen werden. 
Diese Struktur ermöglicht es, dass sämtliche erforderlichen Assets, die für die Ausführung der Anwendung notwendig 
sind, ohne weitere Modifikationen verwendet werden. Die Unity WebGL-Dateien befinden sich dabei im Unterordner 
./unity, welcher die index.html-Datei zusammen mit den dazugehörigen JavaScript- und Mediendateien enthält.
\footnote{Alle Informationen zur Struktur von Angular und zum public-Ordner Folder stammen von \cite{Google}}

Für den Zugriff auf die Unity-Anwendung innerhalb der Angular-Anwendung wird das iframe-Element verwendet. (siehe Listing \ref{lst:unity-iframe-html-code}) 
Das iframe-Tag lädt die index.html-Datei des Unity-Projekts, welche sich im ./unity-Ordner des public-Verzeichnisses 
befindet. Diese Vorgehensweise erlaubt es, die Unity-Anwendung in die Benutzeroberfläche der Angular-Anwendung zu 
integrieren, ohne dass eine vollständige Umstrukturierung oder umfangreiche Anpassungen erforderlich sind. Durch 
die Verwendung von iframes wird der Unity-Inhalt isoliert von der Hauptanwendung geladen, wodurch eine klare 
Trennung zwischen der Angular-Frontend-Anwendung und dem Unity-WebGL-Projekt besteht. Diese Methode hat den 
Vorteil, dass die Unity-Anwendung unabhängig von der Angular-Struktur läuft, was eine einfache Handhabung und 
Wartung des Projekts ermöglicht.
\footnote{Alle Informationen zu Iframes stammen von \cite{MozillaFoundationb}}

\begin{lstlisting}[numbers=left,caption={Iframe-Zugriff auf Unity},label={lst:unity-iframe-html-code}]
<iframe
    width="100%"
    height="700"
    [src]="safeUrl">
</iframe>
\end{lstlisting}

Die safeUrl-Variable (siehe Listing \ref{lst:unity-iframe-html-code}) ist ein SafeResourceUrl-Typ, 
welcher von Angulars DomSanitizer zur sicheren Handhabung von URLs verwendet wird. In der Klasse 
ExploreWorldsPageComponent wird safeUrl als null initialisiert und später gesetzt, sobald ein 
gültiger Token erhalten wird.

Wenn es Änderungen des Tokens gibt, wird eine URL mit dem erhaltenen Token erstellt. Diese URL wird durch den 
DomSanitizer verarbeitet, um sie sicher zu machen, bevor sie in das iframe eingefügt wird. Dabei wird 
überprüft, ob der Token gültig ist. Wenn ja, wird die URL mit dem Token und dem Server-URI gebaut und durch 
bypassSecurityTrustResourceUrl als sichere URL markiert. Diese Methode ist erforderlich, da Angular URLs, 
die in einem iframe eingebettet werden, aus Sicherheitsgründen standardmäßig nicht vertraut. (siehe Listing \ref{lst:url-sanitizer-code})

\begin{lstlisting}[numbers=left,caption={URL Sanitizer},label={lst:url-sanitizer-code}]
    this.token.subscribe(token => {
        let myToken = token.token;
      
        if (myToken !== null && myToken !== undefined && myToken !== "") {
          this.safeUrl = this.sanitizer
            .bypassSecurityTrustResourceUrl(`/unity/index.html?token=${myToken}&server=${environment.apiConfig.uri}`);
        } else {
          this.safeUrl = null;
        }
      });      
\end{lstlisting}


