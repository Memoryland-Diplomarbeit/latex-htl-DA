\setauthor{Isabel Schnalzenberger}

\section{Technologien}

\subsection{Unity}

\textbf{Unity} ist eine Entwicklungsumgebung (\Gls{ide}) für Spiele, auch als \textbf{Spiel-Engine (bzw. Gameengine)} bekannt, die von Unity Technologies mit Hauptsitz in San Francisco entwickelt wurde. Diese vielseitige Plattform ermöglicht die Erstellung von Spielen und interaktiven 3D-Grafik-Anwendungen für verschiedene Zielplattformen, darunter PCs (Windows und Mac), Spielkonsolen, mobile Geräte und Webbrowser.\footnote{Alle Infos zu Unity stammen von: \cite{UnityDocs}}

\begin{figure} [h t]
    \centering
    \includegraphics[scale=0.15]{pics/unity-hub.png}
    \caption{Entwicklung in Unity: der Unity-Hub}
    \label{fig:unity-hub}
\end{figure}


Über den \textbf{Unity-Hub} (siehe \ref{fig:unity-hub}) steigt man in den Unity Editor (die IDE, siehe \ref{fig:unity-ide}) ein. Diese Umgebung ist gängigen 3D-Animationsprogrammen und 3D-Entwicklungs-umgebungen nachempfunden. Die Ansicht in der Mitte stellt die aktuelle Sicht auf die Szene dar. Alle gewählten und in der Szene platzierten Objekte befinden sich links in der Hierarchie. Diese wird vor allem für die Auswahl der zu ändernden Objekte verwendet. Auf der rechten Seite befindet sich der Inspektor. Mit seiner Hilfe verändert man einzelne Attribute der sogenannten GameObjects. Es gibt einfach zu ändernde Attribute und veränderliche bzw. durch Programme veränderte Attribute. Die Kombination aus beidem macht Unity so mächtig.\footnote{Alle Infos zum Unity Editor finden sich auch in \cite{UnityDocsEditor}}


\begin{figure} [h t]
    \centering
    \includegraphics[scale=0.1]{pics/unity-ide.png}
    \caption{Entwicklung in Unity: Die Unity IDE}
    \label{fig:unity-ide}
\end{figure}



Der Entschluss diese Technologie zu verwenden fiel vor allem, weil diese Technologie bereits frühzeitig in unserem Lehrplan enthalten ist und damit umfangreiche Kenntnisse vorhanden waren. Der andere, wesentliche Grund liegt in der \textbf{Erweiterbarkeit} der Umgebung die man in Unity schafft. Die Entwicklungsumgebung ist dergestalt, dass man von Beginn an mit mehreren Scenen im Entwurf gearbeitet hat. Jede Szene (Scene) in Unity spiegelt dabei einen Memorylandtype\footnote{siehe dazu vor allem Abschnitt \ref{sec:memoryland-types}}



\subsection{Visual Studio 2022 und Code}

Der \textbf{Unity Editor} integriert sich mit verschiedenen Source Code Editoren (\Glspl{ide}). Die Entwicklung der entsprechenden Klassen weiter unten erfolgte teils in Windows mit Visual Studio 2022 und später auch unter Ubuntu mit Visual Studio Code und anderen Editoren. Die Editoren unterscheiden sich in der Unterstützung der entsprechenden Intellisense und Syntax-Erkennung. Generell ist aber zu sagen, dass die Entwicklung unter Windows angenehmer ist, da hier die volle Bandbreite einer Umgebung zur Verfügung steht, die einerseits die Bibliotheken kennt und die Syntax prüft.



\section{Unity WebGL}
\label{sec:unity-webgl}

\textbf{Unity} bietet eine besondere Form des Build an: WebGL. \textbf{Unity WebGL} ermöglicht damit die Spiele und interaktiven Anwendungen, die man mit Unity erstellt hat, direkt im Webbrowser aufzurufen, ohne dass zusätzliche Plugins oder Software installiert werden müsste.\footnote{Alle Infos zu Unity WebGL stammen von \cite{UnityDocsWebGL} bzw. \cite{WebGL}}


\begin{wrapfigure}{r}{0.5\textwidth}
    \centering
    \includegraphics[scale=0.12]{pics/unity-bulid-settings.png}
    \caption{Entwicklung in Unity: Die Unity Build Settings}
    \label{fig:unity-build-settings}

    \vspace{1cm}

    \centering
    \includegraphics[scale=0.09]{pics/unity-build-webgl-settings.png}
    \caption{Entwicklung in Unity: Die Unity Build Settings für WebGL}
    \label{fig:unity-build-webgl-settings}
\end{wrapfigure}



\textbf{WebGL (Web Graphics Library)} ist eine JavaScript-API, die 3D-Grafiken im Browser rendern kann. Sie basiert dabei auf OpenGL, einer ``shading language'', welche die Beschreibung einer virutellen Umgebung übernimmt. Unity nutzt diese API, um die für Webbrowser optimierte Ausführung von Inhalten zu ermöglichen. WebGL in Unity unterstützt viele gängige Plattformen wie Desktop-Browser (Chrome, Firefox, Safari, etc.) und erlaubt es diesem Projekt die Anwendung Memoryland schnell und einfach online zu verbreiten.


Das Projekt wurde mit den in den Abbildungen \ref{fig:unity-build-settings} und \ref{fig:unity-build-webgl-settings} dargestellten \textbf{Einstellungen} für Unity WebGL entwickelt und der erzeugte JavaScript Code wurde in das Frontend\footnote{im Verzeichnis ``./unity'' siehe Abschnitt \ref{sec:frontend-integration-webgl}} integriert.


Die gewählten Einstellungen entsprechen den Empfehlungen der bereits erwähnten Beschreibungen zu Unity WebGL und insbesondere das ``Data Caching'' wurde eingestellt. Leider funktioniert in diesem Projekt derzeit das gewählte Caching nicht perfekt. Unity WebGL lädt die Datendatei laufend neu und dies ist laut verschiedenen Quellen im Netz derzeit ein Fehler in der gewählten Umgebung, der in naher Zukunft behoben werden soll.\footnote{Hauptquellen: \cite{UnityDocsDataCachingIssue} und \cite{UnityDocsDataCachingIssue2}}


\section{Implementierung des Unity Projekts}

Die Entwicklung des \textbf{Unity Projektes} ist der grö\ss{}te Teil der Frontend Entwicklung. Die Umsetzung enthielt einerseits den Datenaustausch zwischen Unity und dem Frontend, Unity und dem Backend, Unity und dem Azure Storage und zu guter Letzt, die Darstellung der gewünschten Szenen. Die Memorylands sollten detailreich aber nicht langsam, ansehnlich aber nicht zu realistisch und auch einfach zu bedienen sein. Wenn der Nutzer ein Memoryland seinem digital-ungeübten Verwandten schickt, soll dieser sich einfach zurücklehen können und das Erlebnis genie\ss{}en.



\subsection{Url-Parameter}
\label{subsec:unity-url-parameter}

Es ist notwendig \emph{Daten vom Frontend mit Unity} auszutauschen. Es gibt derzeit viele entsprechende Mechanismen. Die gewählte \textbf{Methode über URL-Parameter} vermeidet Verschränkungen zwischen der Implementierung des Frontend (und dessen Services in Richtung Backend) und der Implementierung von den Szenen in Unity.


Für den gewählten Prozess der \emph{Übergabe der Parameter} über den IFrame mit Hilfe von herkömmlichen URL-Parametern (siehe Listing \ref{lst:unity-url-parameter-in-action}) gab es bereits mehrere Realisierungen und Erfahrungsberichte. Diese verwendeten alle eine Form der Verschränkung von JavaScript Code mit Unity C\# Code. 



\begin{lstlisting}[numbers=left,caption={IFrame for Unity in Frontend},label={lst:unity-url-parameter-in-action}]
this.safeUrl = this.sanitizer
          .bypassSecurityTrustResourceUrl(`/unity/index.html?token=${myToken}& server=${environment.apiConfig.uri}`);
\end{lstlisting}


Es wurde eine fortgeschrittene Entwicklung einer \textbf{UrlParameter}-Klasse aus dem Projekt übernommen und erweitert.\footnote{Der Originalcode wurde von \cite{UrlParameter} übernommen und um eine weitere Funktion erweitert.} Die enthaltenen Funktionen sahen nur nummerische Parameter vor und so war die gewählte Übergabe eines GUID-Token aus dem Frontend vorerst nicht möglich. Die Funktion ``GetString'' (siehe Listing \ref{lst:unity-url-parameter}) wurde daher in die Klasse \textbf{``DictionaryStringStringExt''} eingefügt. Es waren ansonsten keine weiteren Änderungen notwendig.

Das vorhandene \textbf{Singleton-Skript} ermöglicht einen \textbf{einfachen Zugriff} auf alle URL-Parameter und -Komponenten in einer WebGL-Build-Anwendung. Es muss hierzu eine jslib-Datei verwendet werden, die den JavaScript-Teil aktiviert. Dieses Skript wird beim Importieren in ein Projekt automatisch eine jslib-Datei in den Plugins-Ordner extrahieren, die den Namen „URLParameters.jslib“ trägt.

Um auf die verschiedenen URI-Teile zuzugreifen, kann man einfach die statischen Methoden dieser Klasse verwenden. Die verfügbaren Eigenschaften sind Protocol, Hostname, Port, Pathname, Search, Hash sowie Host, Origin und Href. Damit kann man mithilfe der Funktion \textbf{``URLParameters.GetSearchParameters()''} an beliebiger Stelle in Unity ein Dictionary mit den Url-Parametern laden.

\begin{lstlisting}[numbers=left,caption={UrlParameter},label={lst:unity-url-parameter}]
public static class DictionaryStringStringExt
{
    public static double GetDouble(this Dictionary<string, string> aDict, string aKey, double aDefault)
    {
        string str;
        if (!aDict.TryGetValue(aKey, out str))
            return aDefault;
        double val;
        if (!double.TryParse(str, NumberStyles.Float, CultureInfo.InvariantCulture, out val))
            return aDefault;
        return val;
    }
    public static int GetInt(this Dictionary<string, string> aDict, string aKey, int aDefault)
    {
        string str;
        if (!aDict.TryGetValue(aKey, out str))
            return aDefault;
        int val;
        if (!int.TryParse(str, NumberStyles.Integer, CultureInfo.InvariantCulture, out val))
            return aDefault;
        return val;
    }
    public static string GetString(this Dictionary<string, string> aDict, string aKey, string aDefault)
    {
        string str;
        if (!aDict.TryGetValue(aKey, out str))
            return aDefault;
        return str;
    }
}
\end{lstlisting}


\subsection{Die Startup Szene}
\label{subsec:unity-startup-scene}


\begin{figure} [h t]
    \centering
    \includegraphics[scale=0.12]{pics/unity-ui-builder.png}
    \caption{Entwicklung in Unity: Der Unity UI Toolkit}
    \label{fig:unity-ui-builder}
\end{figure}


Die Szene ist bereits in Abbildung \ref{fig:unity-ide} zu sehen. Es wurde ein \textbf{leeres Terrain} gebaut und eine \textbf{Unity UI} darübergelegt.\footnote{Die dazu verwendete Technologie Unity UI Toolkit ist in \cite{UnityDocsUIToolkit} beschrieben.} Diese UI dient dazu den Nutzer zu informieren, dass nun die Bilder und die Szene geladen werden. Im Listing \ref{lst:unity-startup-document} sind die wesentlichen Codeteile enthalten. Die Methode \textbf{``OnEnable''} wird aufgerufen, wenn die Startup-Szene die UI aufruft, also zu Beginn des Memoryland. Es werden die UI Komponenten in die Membervariablen geladen und danach sofort die asynchrone Routine zur Überwachung der Fotobeladung im \textbf{``SceneChanger''} gestartet. 


Die Während der ``SceneChanger'' (siehe Listing \ref{lst:unity-scene-changer-start}) die Beladung der Texturen betreibt, wartet der \textbf{``StartupDocument''-UI} bereit darauf, dass es diese Aufgabe abschlie\ss{}t. Diese Information wird mittels der asynchronen Funktion \textbf{``isValid''} (siehe Listing \ref{lst:unity-scene-changer-public}) überprüft. Sobald die Daten und Fotos geladen sind, steht dort der Szenenname zur Verfügung. Falls nun die Szene bekannt ist (derzeit nur ``forest'' oder ``island''), kann die Szene gestartet werden, ansosnten wird der Benutzer mittels des Informationsfeldes darauf hingewiesen, dass ein Fehler aufgetreten ist.



\begin{lstlisting}[numbers=left,caption={StartupDocument für Unity UI Toolkit},label={lst:unity-startup-document}]
....    
    public class StartupDocument : MonoBehaviour
    {
        private Button _start;
        private Label _loading;
        private string chosenScene = null;
           
        //Add logic that interacts with the UI controls in the `OnEnable` methods
        private void OnEnable()
        {
            // The UXML is already instantiated by the UIDocument component
            var uiDocument = GetComponent<UIDocument>();
    
            _start = uiDocument.rootVisualElement.Q("start") as Button;
            _start.SetEnabled(false);
            _loading = uiDocument.rootVisualElement.Q("loading") as Label;
            _start.RegisterCallback<ClickEvent>(PrintClickMessage);
            
            StartCoroutine(LoadFromLikeCoroutine());
        }
    
        private IEnumerator LoadFromLikeCoroutine()
        {
            while (chosenScene == null) {
            IEnumerator tgt = FindAnyObjectByType<SceneChanger>().IsValid();
            while (tgt.MoveNext()) 
            {
                chosenScene = (string)tgt.Current;
                yield return chosenScene;
            }
            }
            
            if (chosenScene != "ERROR") {
                Debug.Log("Finished!");
                _loading.text = "Finished Loading " + chosenScene;
                _start.SetEnabled(true);
                FindAnyObjectByType<SceneChanger>().StartThisNow();
            }
            else 
            {
                _loading.text = "Sorry, ERROR Loading this Memoryland. Possibly Token invalid!";
                Debug.Log("ERROR loading");
            }
        }
        ....
    }        
\end{lstlisting}


Die Klasse \textbf{``SceneChanger''} existiert länger im Projekt. Zu Beginn hatte sie nur die Aufgabe den Szenenwechsel bereitzustellen, inzwischen hat sich das Aufgabengebiet auf das Laden aller Bestandteile der Szenen ausgeweitet. Sie hat daher drei Aufgaben: das \textbf{Laden der Memoryland-Informationen} aus dem Backend, die Beladung der Texturenliste (also der \textbf{Fotos}) und die Bereitstellung einer \textbf{Szenen-Wechsel-Funktion}.

Die Funktion \textbf{``Start''} lädt die URL-Parameter mit Hilfe der Klasse aus Abschnitt \ref{subsec:unity-url-parameter} in die enstprechenden statischen Membervariablen. Zu Testzwecken wurde ein ``fallbackResponse'' und ein dazugehöriger Parameter (``fallback'' == ``yes'') eingeführt, sodass Szenen zeitnah und während der Implementierung mit echten Bildern aus dem Backend getestet werden können.

Nach der Beladung der Variablen werden diese verwendet, um die URL zum \textbf{Aufruf des REST Backend} mit dem Token\footnote{siehe dazu vor allem auch die Abschnitts \ref{backend-memoryland-controller} und \ref{subsec:frontend-explore-worlds}} zusammenzusetzen. Die Verwendung des Token ermöglicht den Aufruf der Memorylands auch durch nicht-eingeloggte Benutzer.



\begin{lstlisting}[numbers=left,caption={SceneChanger - Start},label={lst:unity-scene-changer-start}]
        ....
        
        public class SceneChanger : MonoBehaviour
        {
        
            private static 
                string fallbackResponse = @"
                ....
                ";
        
            private static string chosenScene = null;
            private static string token = null;
            private static string server = "https://app-memoryland.azurewebsites.net";
            private static string fallback = "no";
            private static Dictionary<int, string> allUrls = null;
        
            private static Dictionary<int, Texture> allTextures = new Dictionary<int, Texture> ();
        
            ....                
        
            // Start is called before the first frame update
            IEnumerator Start()
            {
                // already loaded?
                if (SceneChanger.token == null && SceneChanger.chosenScene == null) 
                {
                    // first load the given token
                    Dictionary<string,string> paras =  URLParameters.GetSearchParameters();
                    SceneChanger.token = paras.GetString("token", "");
                    SceneChanger.server = paras.GetString("server", "https://app-memoryland.azurewebsites.net");
                    
                    if (Application.platform == RuntimePlatform.WebGLPlayer)
                    {
                        SceneChanger.fallback = paras.GetString("fallback", "no");
                    }
                    else
                    {
                        SceneChanger.fallback = paras.GetString("fallback", "no");
                        SceneChanger.token = paras.GetString("token", "1920cf9e-b295-4c80-b347-75eff21a71f6");
                    }
                    
                    Debug.Log("Parameter Value token = " + SceneChanger.token);
                    Debug.Log("Parameter Value server = " + SceneChanger.server);
                    Debug.Log("Parameter Value fallback = " + SceneChanger.fallback);
                    
                    if (SceneChanger.fallback == "yes")
                    {
                        yield return SetInternals(SceneChanger.fallbackResponse);
                        Debug.LogError("Fallback is set => Using simulation init parameters to start with!");
                    }
                    else
                    {
                        // missing call to the url/backend
                        string uri = SceneChanger.server+"/api/Memoryland?token=" + SceneChanger.token;
                        string response = "";
                        
                        using (UnityWebRequest webRequest = UnityWebRequest.Get(uri))
                        {
                            yield return webRequest.SendWebRequest();
                        
                            switch (webRequest.result)
                            {
                            case UnityWebRequest.Result.ConnectionError:
                            case UnityWebRequest.Result.DataProcessingError:
                                Debug.LogError(SceneChanger.server + ": Error: " + webRequest.error);
                                    SceneChanger.chosenScene = "ERROR";
                                break;
                            case UnityWebRequest.Result.ProtocolError:
                                Debug.LogError(SceneChanger.server + ": HTTP Error: " + webRequest.error);
                                    SceneChanger.chosenScene = "ERROR";
                                break;
                            case UnityWebRequest.Result.Success:
                                Debug.Log(SceneChanger.server + ": Received: " + webRequest.downloadHandler.text);
                                response = webRequest.downloadHandler.text;
                                yield return SetInternals(response);
                                break;
                            default:
                                Debug.LogError(SceneChanger.server + ": unknown " + webRequest.result.ToString());
                                    SceneChanger.chosenScene = "ERROR";
                                break;
                            }
                            }
                        }
                }		
            }
        ....
        }
\end{lstlisting}
    

Das Umwandeln des gelieferten JSON im Textformat übernimmt dann die Funktion \textbf{``SetInternals''}. Es wurden einige serialisierbare Basisklassen für die im JSON enthaltenen Daten erstellt. Diese sind ``MemorylandType'', ``MemorylandConfiguration'' und ``Memoryland''. Die Funktion nimmt den gelieferten String und verwendet die Unity-Basisklasse \textbf{``JsonUtility''} um diesen zu C\# Objekten zu konvertieren.

Danach wird sofort für jede gelieferte Foto-URL die asynchrone Funktion \textbf{``LoadTexture''} aufgerufen. Dies ermöglicht ein möglichst rasches Laden der Bilder bei einem verzerrungs- und störungsfreien Ansehen des Memorylands bzw. der entsprechenden Szene. 

\begin{lstlisting}[numbers=left,caption={SceneChanger - SetInternals},label={lst:unity-scene-changer-set-internals}]
    ....   
    
        [System.Serializable]
        public class MemorylandType
        {
            public string name;
            public int photoAmount;
        }
        
        [System.Serializable]
        public class MemorylandConfiguration
        {
            public int position;
            public string photo;
        }
        
        
        [System.Serializable]
        public class Memoryland
        {
            public long id;
            public string name;
            public MemorylandType memorylandType;
            public List<MemorylandConfiguration> memorylandConfigurations;
        
            public static Memoryland CreateFromJSON(string jsonString)
            {
                return JsonUtility.FromJson<Memoryland>(jsonString);
            }
        }
    ....
    private IEnumerator SetInternals(string response) 
    {
        Debug.Log("setInternals now with ");
        if (response != null) {
        
            Memoryland ml = null;
            try {
                ml = Memoryland.CreateFromJSON(response);
            }
            catch (Exception e)
            {
                Debug.LogError("Error Reading response! " + e.ToString());
            }
            
            if (ml != null) 
            {
                if (ml.memorylandConfigurations != null) {
                    SceneChanger.allUrls = new Dictionary<int, string>();
                    foreach (MemorylandConfiguration item in ml.memorylandConfigurations) {
                        Debug.Log("got one photo " + item.position.ToString());
                        SceneChanger.allUrls[item.position] = item.photo;
                        yield return LoadTexture(item.position, item.photo);
                    }
                }

                // always set the pics first
                if (ml.memorylandType != null && ml.memorylandType.name != null ) {
                    SceneChanger.chosenScene = ml.memorylandType.name;
                    Debug.Log("Got Scene from response: " + chosenScene);
                }
            }
        }
        
        if (SceneChanger.allUrls == null) 
        {
            SceneChanger.chosenScene = "ERROR";
            Debug.LogError("no URLs loaded!");
        }
    }    
\end{lstlisting}
    
Die Bilder bzw. Fotos werden vom Backend in Form von URLs mit SAS-Tokens (mit beschränkter Gültigkeit\footnote{siehe dazu vor allem Abschnitt \ref{subsection:sas-token-generation}}) beim Call gegen das REST Backend geliefert. Diese Bilder werden sofort als Texturen in den Speicher geladen um Verzögerungen und fehlende Darstellungen zu vermeiden. Dies übernimmt die in Listing \ref{lst:unity-scene-changer-load-texture} enthaltene Funktion \textbf{``LoadTexture''}. Es wird dazu - wie oben für den Call ins Backend - die Unity Basisklasse \textbf{``UnityWebRequest''} verwendet.


\begin{lstlisting}[numbers=left,caption={SceneChanger - LoadTexture},label={lst:unity-scene-changer-load-texture}]
  
    IEnumerator LoadTexture (int id, string uri)
    {
        Debug.Log("Load Photo " + id.ToString() + " with " + uri);
        Texture myTexture = null;
        using (UnityWebRequest imageRequest = UnityWebRequestTexture.GetTexture(uri))
        {
            yield return imageRequest.SendWebRequest();
            
            switch (imageRequest.result)
            {
            case UnityWebRequest.Result.ConnectionError:
            case UnityWebRequest.Result.DataProcessingError:
                Debug.LogError("Photo Error: " + imageRequest.error);
                break;
            case UnityWebRequest.Result.ProtocolError:
                Debug.LogError("Photo HTTP Error: " + imageRequest.error);
                break;
            case UnityWebRequest.Result.Success:
                Debug.Log("Received Photo " + id.ToString() + " with " + uri);
                //
                myTexture = DownloadHandlerTexture.GetContent(imageRequest);
                SceneChanger.allTextures[id] = myTexture;
                break;
            default:
                Debug.LogError("Photo unknown result: " + imageRequest.result.ToString());
                break;
            }	    
        }
            yield return "Loaded";
    }
\end{lstlisting}
    

Schlussendlich bleibt die letzte Funktion des SceneChanger, welche schon von Beginn bestanden hatte: der Szenenwechsel mithilfe von \textbf{``StartThisNow''}. Sie wird mit Hilfe der Standard-Unity Funktionalität von SceneManager abgebildet. Es wird zusätzlich überprüft, ob diese Szene überhaupt im WebGL Build existiert.\footnote{Die Build-Settings von WebGL können der Abbildung \ref{fig:unity-build-settings} entnommen werden. Dort steht auch eine Liste der enthaltenen Szenen.} Die weiteren Funktionen \textbf{``isValid'' und ``GetOneTexture''} werden von anderen Klassen verwendet um die aktuelle Szene bzw. einzelne Fotos (bereits als Texturen) abzurufen.


\begin{lstlisting}[numbers=left,caption={SceneChanger},label={lst:unity-scene-changer-public}]
....
    public class SceneChanger : MonoBehaviour
    {
        ....
        
        private void ChangeScene(string sceneName)
        {
            if (SceneManager.GetActiveScene().name == sceneName + "Scene" || (sceneName == "ERROR"))
            {
                Debug.Log("Scene " + sceneName + " already active!");
            }
            else 
            {
                Debug.Log("Change Scene to " + sceneName);
                SceneManager.LoadScene (sceneName + "Scene");
            }
        }
        ....
        public IEnumerator GetOneTexture(int position)
        {
            if (SceneChanger.allUrls == null || SceneChanger.chosenScene == null) // not finished call to backend
                yield return null;
                
            if (SceneChanger.allTextures.ContainsKey(position)) {
                yield return SceneChanger.allTextures[position];
            }
            else 
            {
                Debug.Log(position.ToString() + " NOT FOUND in MemoryLand!");
                yield return null;
            }
        }
    
        ....    
        public IEnumerator IsValid()
        {
            if (SceneChanger.chosenScene == null) // not finished call to backend
                yield return null;
                
            yield return (SceneChanger.chosenScene);
        }
    
        public void StartThisNow()
        {
            // now change scene
            ChangeScene(SceneChanger.chosenScene);
        }
    
        ....
    }
\end{lstlisting}

\subsection{Einfügen von Images}
\label{subsec:unity-image-loader}

Am Beginn des Projektes wurden die Fotos als Texturen vom Backend erst nach dem Start der Szene geladen, also in Form eines Lazy-Load während die Szene bereits abläuft. Dies führte dazu, dass die ersten Fotos noch nicht fertig geladen waren und die Planes noch im Platzhalter-Modus (hier wurde eine blaue Farbe für das Plane gewählt) dargestellt wurden. Dieses Vorgehen war äußerst unpassend für das angestrebte Nutzererlebnis. 

Das Laden der Fotos wurden daher im Rahmen einem weiteren Sprint im Code zeitlich vorverlegt und findet nun während des Startup in der Startup-Szene statt. Dazu wird der Nutzer im Rahmen der Startup-Szene (siehe Abschnitt \ref{subsec:unity-startup-scene}) im gewählten UI über den Ladeprozessfortschritt informiert.

Der hier beschriebene \textbf{``ImageLoader''} hat damit nur noch die Aufgabe die entsprechende Texture vom \textbf{``SceneChanger''} (siehe Abschnitt \ref{subsec:unity-startup-scene}) zu laden und dann das Format des Plane anzupassen. Die Bilder würden ansonsten verzerrt angezeigt. Dazu wird mit Hilfe eines \textbf{``Vector3''} Objekts die Größe und damit Verzerrung des Planes eingestellt. Aufrecht stehende Bilder können in Unity nicht gedreht werden (EXIF-Informationen werden ignoriert), daher musste dies bereits vom Backend übernommen werden.\footnote{siehe hierzu vor allem Abschnitt \ref{subsec:backend-rotate-pictures}} 


\begin{lstlisting}[numbers=left,caption={ImageLoader},label={lst:unity-image-loader}]
....
public class ImageLoader : MonoBehaviour
{
    public int urlID = 0;
    public Renderer thisRenderer;

    // Start is called before the first frame update
    void Start()
    {
        // the following will be called even before the load finished
        thisRenderer.material.color = Color.blue;

        StartCoroutine(LoadFromLikeCoroutine()); // execute the section independently
    }

....

    // this section will be run independently
    private IEnumerator LoadFromLikeCoroutine()
    {
	Texture myTexture = null;
	
        IEnumerator tgt = FindAnyObjectByType<SceneChanger>().GetOneTexture(urlID);
        while (tgt.MoveNext()) 
        {
        	myTexture = (Texture)tgt.Current;
        	yield return myTexture;
        }
        
        if (myTexture == null) {
        	yield return null;
        }

	if (myTexture == null) {
            	thisRenderer.material.color = Color.red;          // set red
		Debug.Log("Image NotFound!");
		yield return null;
	}
	else 
	{
		Debug.Log("Loaded");
		thisRenderer.material.color = Color.white;          // set white

		float scale = 0.3f;

		//thisRenderer.gameObject.
		if (myTexture.height > myTexture.width)
		{
			float f = (float)myTexture.height / (float)myTexture.width * scale;
			thisRenderer.transform.localScale = new Vector3(scale, scale, f);
			//thisRenderer.material.SetTextureScale("_MainTex", new Vector2(1f, 0.5f));

		}
		else
		{
			//float f = ((float)myTexture.width * (float)0.5) / ((float)myTexture.height * (float)0.5);
			float f = (float)myTexture.width / (float)myTexture.height * scale;
			thisRenderer.transform.localScale = new Vector3(f, scale, scale);
			//thisRenderer.material.SetTextureScale("_MainTex", new Vector2(0.5f, 1f));

		}

		thisRenderer.material.mainTexture = myTexture ;  // set loaded image
	}
    }
}
\end{lstlisting}




\section{Erstellung von Szenen bzw. Memoryland-Typen}

Das Projekt enthält bereits zwei vordefinierte Szenen:

\begin{itemize}
    \item Die \textbf{Wald-Szene} (MemorylandType == ``forest'') mit 10 Bildern
    \item Die \textbf{Insel-Szene} (MemorylandType == ``island'') mit 10 Bildern
\end{itemize}


\subsection{Der Dolly Track}
\label{subsec:unity-dolly-track}

Ein Dolly Track ist ein Hilfsmittel, um den automatischen Durchlauf durch die Szene zu gestalten. Der gewählte Durchlauf bei der Insel-Szene kann in der Abbildung \ref{fig:unity-dolly-track} betrachtet werden.


\begin{figure} [h t]
    \centering
    \includegraphics[scale=0.15]{pics/unity-dolly-track.png}
    \caption{Entwicklung in Unity: Die Insel Szene}
    \label{fig:unity-dolly-track}
\end{figure}



..... \emph{OFFEN} WEITER, TODO





\subsection{Die Wald-Szene}

Die Erstellung der Wald-Szene war am schwierigsten, da die ersten Entwürfe und die Ideenfindung über diese Szene erfolgte. Es wurde versucht einerseits eine \textbf{ansprechende und umfangreiche Waldumgebung} zu erzeugen. Die Idee, die hinter einer Waldumgebung stand, ist die Darstellung von Fotos und Erinnerungen aus einer Wanderung im Grünen.



\begin{figure} [h t]
    \centering
    \includegraphics[scale=0.15]{pics/unity-edit-forest.png}
    \caption{Entwicklung in Unity: Die Wald Szene}
    \label{fig:unity-edit-forest}
\end{figure}


Diese \textbf{Wanderungserinnerung} ließ sich für die Autoren am klarsten durch einen Wald abrufen. Ein Wald mit vielen verschiedenen Bäumen und Blumen. Die Szene wurde also entsprechend aufgebaut. Dabei wurde zuerst teilweise auf Beispiele aus dem Unterricht zurückgegriffen. Diese waren aber leider \textbf{zu detailreich}, um auch wirklich auf allen Zielplattformen dargestellt werden zu können. Die Anzahl der Details wurde daher reduziert.

Die Wanderung selbst wird durch die Verwendung des Konzepts des \textbf{Dolly Track} (siehe Abschnitt \ref{subsec:unity-dolly-track}) durchgeführt. Die Nutzer erleben einen Waldspaziergang mit Bäumen und Blumen, während sie an den Bildern vorüberschreiten.


\subsection{Die Insel-Szene}

Nachdem die umfangreichen Tests der Integration der Unity WebGL in das Frontend und das laufende Backend erfolgreich waren, stand nun der Weg offen für die Entwicklung einer \textbf{weiteren Ferien-Umgebungs-Szene}. Die \textbf{Insel-Szene} war mit dem Erlebnis einer \textbf{sandigen Insel mit Sonne, Strand und Meer bei Sonnenuntergang} für die Entwickler die logischste Wahl.   


\begin{figure} [h t]
    \centering
    \includegraphics[scale=0.15]{pics/unity-edit-island.png}
    \caption{Entwicklung in Unity: Die Insel Szene}
    \label{fig:unity-edit-island}
\end{figure}


Mehrere Palmen (``Palm Trees'') und Steine wurden auf der Insel in verschiedenen Winkeln und Punkten so platziert, dass der Eindruck einer Südseeinsel entsteht. Die Sonne wurde so platziert, dass es sich um einen Sonnenauf- bzw. -untergang handeln könnte. Einige kleinere Elemente, wie bspw. dürre Gräser wurden in der nähe des Meeres platziert. Das Meer selbst ist auch ein umfangreiches Prefab namens ``Detailed Water''. 


\begin{wrapfigure}{r}{0.5\textwidth}
    \centering
    \includegraphics[scale=0.15]{pics/unity-island-light-setting.png}
    \caption{Die Panels und die Spotlights in der Hierarchie}
    \label{fig:unity-island-light-setting}

    \vspace{1cm}

    \centering
    \includegraphics[scale=0.15]{pics/unity-island-light-setting-detail.png}
    \caption{Spezielle Einstellungen für die Spotlights}
    \label{fig:unity-island-light-setting-detail}
\end{wrapfigure}



Die Wanderung selbst wird durch die Verwendung des Konzepts des Dolly Track (siehe Abschnitt \ref{subsec:unity-dolly-track}) durchgeführt. Die Nutzer erleben einen Inselspaziergang mit Sonne, Strand und Meer, während sie an den Bildern ihrer Erinnerungen vorüberschreiten.



Durch das spezielle Setting im Sonnenuntergang war es zusätzlich notwendig die Bilder \textbf{extra zu beleuchten}. Diese Beleuchtung wurde durch einzelne Spotlights realisiert. Die Planes für die Bilder wurden einer eigenen Layer hinzugefügt und die Spotlights beleuchten nur Elemente dieser Layer. Die beiden Abbildungen \ref{fig:unity-island-light-setting} und \ref{unity-island-light-setting-detail} zeigen die dafür gewählten Einstellungen.


\subsection{Erweiterbarkeit}
\label{subsec:unity-erweiterbarkeit}

Nach dem vorliegenden Konzept kann nun um beliebige Szenen erweitert werden. Die Szenen müssen mehreren Entwicklungsanforderungen und Grundkonzepten folgen. Die größte Freiheit besteht jedoch darin, dass die Szenen zukünftig auch so programmiert werden können, dass man sich beispielsweise selbständig durch die Szene beweigen könnte.

Erste Voraussetzung ist die \textbf{Darstellung der Bilder} mit Hilfe der bereits erwähnten Planes. Diese Planes verwenden das Script ``ImageLoader''\footnote{siehe dazu \ref{subsec:unity-image-loader}}, um die dann bereits geladenen Texturen aus dem SceneChanger zu holen. Damit ist der Aufwand gering, um mehr oder auch weniger Bilder in einer Szene zu integrieren.

Die zweite Voraussetzung ist die Integration der Startup


..... \emph{OFFEN} WEITER, TODO

