
\section{Entstehung der Projektidee}

Die Idee für das spätere Diplomarbeitsprojekt entstand Mitte des 3. Jahrgangs 
und basierte auf einer persönlichen Erfahrung. Ein Traum diente als zentrale 
Inspirationsquelle: In diesem Traum bewegte sich eine Person durch eine Allee, 
in der zahlreiche Bilder und Videos frei in der Luft schwebten. Diese visuelle 
Darstellung weckte das Interesse an der Möglichkeit, Medieninhalte auf innovative 
Weise zu präsentieren.

Die Entscheidung, diesen Traum schriftlich festzuhalten, wurde durch eine 
Unterrichtssituation beeinflusst. An diesem Tag informierte eine Lehrkraft 
über die Diplomarbeit, die im darauffolgenden Schuljahr zu beginnen war. Während 
dieses Vortrags entstand die erste Skizze des Traums, um die Idee visuell zu erfassen 
und mögliche gestalterische sowie technische Umsetzungsmöglichkeiten zu überdenken.

Nach der Schule wurde die Idee mit einer weiteren Person aus dem familiären Umfeld 
besprochen. Dabei wurde insbesondere die Möglichkeit in Betracht gezogen, das Thema 
der Diplomarbeit auf Basis dieses Konzepts zu entwickeln. Eine weitere Überlegung war, 
ob eine Zusammenarbeit mit dem jetzigen anderem Teammitglied sinnvoll wäre.

In den darauffolgenden Stunden erfolgte eine detaillierte Planung. Dabei wurde ein 
erster Projektantrag verfasst, in dem mögliche technische und konzeptionelle Ansätze 
definiert wurden. Während dieser Phase wurde auch erörtert, welche Ressourcen 
erforderlich wären und welche Herausforderungen bei der Umsetzung zu erwarten 
seien.

Nach der Fertigstellung des ersten Entwurfs des Projektantrags begann die Suche nach 
einer geeigneten Betreuungsperson für die Diplomarbeit. Aufgrund einer Empfehlung von 
Frau Professorin Engleiter Patricia fiel die Wahl auf Herrn Professor Aberger 
Christian. Diese Entscheidung wurde getroffen, da er als fachlich geeignet und 
erfahren für den Bereich, in der die Diplomarbeit fällt, angesehen wurde.

Am Ende des 3. Jahrgangs wurde der Projektantrag Herrn Professor Aberger Christian 
übergeben. Er nahm den Antrag entgegen und empfahl, zu Beginn des folgenden Schuljahres 
erneut Kontakt aufzunehmen, um offene Fragen zu klären und die Projektbetreuung 
endgültig zu bestätigen.


\section{Entstehung des Logos}


\begin{wrapfigure}{r}{0.4\textwidth}
    \centering
    \includegraphics[scale=0.1]{pics/logo_skizzen.png}
    \caption{Erste Skizzen des Logos}
    \label{fig:logo-skizzen}
\end{wrapfigure}

Im Rahmen des Unterrichts, des 4. Jahrgangs, wurde eine gestalterische Aufgabe 
gestellt, die sich mit der Entwicklung eines individuellen Logos befasste. Diese 
Aufgabe war Bestandteil des Unterrichts und wurde während der regulären Unterrichtszeit 
bearbeitet. 


\begin{wrapfigure}{r}{0.3\textwidth}
    \centering
    \includegraphics[scale=0.2]{pics/Farbige Skizze.png}
    \caption{Farbige Skizze des Logos}
    \label{fig:logo-farbige-skizze}
\end{wrapfigure}

Die Aufgabenstellung sah vor, ein eigenes Logo zu entwerfen. Dabei bestand die 
Möglichkeit, ein Logo für ein bereits bestehendes oder geplantes eigenes Projekt 
zu gestalten oder ein völlig freies Design ohne spezifischen Verwendungszweck zu 
entwickeln. Der kreative Prozess umfasste zunächst die Anfertigung zahlreicher 
Skizzen, um verschiedene gestalterische Ansätze zu erproben.

Im Laufe des Schuljahres wurden verschiedene Entwürfe erstellt, überarbeitet und 
weiterentwickelt. Die endgültige Version des Logos wurde gegen Ende des Jahres 
fertiggestellt. Das Endprodukt nahm einer Karte in Form eines Hirnes an, bei der 
Sterne als zentrales Designelement verwendet wurden.



\section{Ausgangssituation}

Herkömmliche Familien-/Fotoalben stehen normalerweise wegen ihres Gewichtes 
zuhause und falls man dann einmal einem Freund bei einer Party ein Foto schnell 
zeigen möchte, hat man eher das Handy als ein ganzes Fotoalbum dabei.

Zwar gibt es schon Tools, welche die Fotos nur präsentieren, 
aber wir wollen die Fotos zeitgemä\ss{} für jeden leicht verfügbar und transportabel animieren.

\section{Untersuchungsanliegen}

\subsection{\firstauthor}

Die vorliegende Untersuchung zielt darauf ab, die effiziente Speicherung 
umfangreicher Mengen von Videos und Bildmaterial in Cloud-Umgebungen zu 
untersuchen sowie die Prozesse zur Erstellung und Bearbeitung von Videos 
auf der Backend-Ebene zu erforschen.

\subsection{\secondauthor}

Die vorliegende Untersuchung zielt darauf ab, die potenzielle Steigerung 
der Akzeptanz von Online-Darstellungen durch die Integration von 
3D-Visualisierungen einer Bildergalerie zu erforschen.

