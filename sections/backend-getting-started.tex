\subsection{Einrichtung des Azure Blob Storage}
\label{subsection:azure-blob-storage-getting-started}

Um die Effizienz bei häufigem Zugriff auf die Bilder möglichst effizient zu gestalten, 
wurden im Azure-Portal die folgenden Einstellungen konfiguriert.
\footnote{Infos dazu, was Azure Blob Storage ist und anderes stammen von hier: \cite{MicrosoftCorporationd}}

\subsubsection{SAS-Tokens}

Um den Zugriff auf Bilder über Shared Access Signatures (SAS) zu ermöglichen, muss in 
die Konfiguration ``Allow storage account key access'' aktiviert werden.

SAS-Tokens ermöglichen es uns, bestimmte Zugriffsrechte zu definieren, wie zum Beispiel
nur ein Lesezugriff. Das ermöglicht es unserem System, mittels Unity direkt auf die
gespeicherten Bilder zuzugreifen.

Mithilfe von SAS-Tokens die im Backend generiert und an Unity übergeben werden, 
kann Unity nun auf die benötigten Bilder zugreifen, ohne vollständige Zugriffsdaten 
des Storage Accounts zu benötigen. 

SAS-Tokens haben eine begrenzte Gültigkeitsdauer, was aus Sicherheitsgründen sinnvoll ist, 
da so der Zugriff auf Ressourcen zeitlich eingeschränkt wird. Das Backend übernimmt hierbei
die erneute Generierung dieser Tokens, sobald sie benötigt werden. Dadurch wird sichergestellt, 
dass autorisierte Nutzer:innen weiterhin auf die Ressourcen zugreifen können, ohne dass 
langfristig gültige Tokens ein Sicherheitsrisiko darstellen.
\footnote{Für Infos, wie Tokens generiert werden, siehe Kapitel \ref{subsection:sas-token-generation}}
\footnote{Alle Informationen zu SAS-Tokens stammen von: \cite{MicrosoftCorporationa}}


\subsubsection{Blob Access Tier}

Um den häufigen Zugriff auf Bilder effizienter zu gestalten, sollte in der Konfiguration 
des Blob Storage der Access Tier die Standardstufe auf ``Hot'' gesetzt werden.

Ein Azure Blob Storage bietet verschiedene Speicherstufen, wie ``Hot'', ``Cool'', ``Cold'',
oder ``Archive''. Cool ist zum Beispiel geeignet für weniger häufige Zugriffe, hat
niedrigere Speicherkosten, dafür aber höhere Zugriffskosten. 

Für uns ist Hot geeignet, da mit dieser Einstellung die Bilder für den häufigen 
Zugriff mit schnelleren Abrufzeiten optimiert werden und die Zugriffskosten niedriger sind. 
Dies führt allerdings zu höheren Speicherkosten. 
\footnote{Alle Informationen zu Blob Access Tiers stammen von \cite{MicrosoftCorporationb}}


\subsubsection{CORS-Header}

CORS (Cross-Origin Resource Sharing) ist eine Sicherheitsfunktion moderner Webbrowser, 
die den Zugriff auf Ressourcen zwischen verschiedenen Ursprüngen (Domains) einschränkt.
Da Memoryland und Unity Zugriff auf Bilder aus dem Azure Blob Storage benötigen, müssen
die folgenden CORS-Einstellungen für ``Blob-Services'' benötigt.

Es müssen die Spalten ``Allowed Origins'', ``Allowed Headers'' und ``Exposed Headers''
mit dem ``*'' Zeichen gefüllt werden, und Als ``Allowed Methods'', sollen ``Head'' und
``GET'' angegeben werden. Dies erlaubt den Zugriff von beliebigen Ursprüngen, mit beliebigen
Headern in der Anfrage. Weiters wird mit ``Exposed Headers'' eingestellt, dass alle 
Antwort-Header für den Client sichtbar sind und die ``Allowed Methods'' beschränkt die 
zulässigen HTTP-Methoden auf das Abrufen von Metadaten (HEAD) und das 
Laden von Bildern (GET). ``Max Age: 0'', verhindert das Caching der CORS-Vorgaben 
durch den Browser, sodass jede Anfrage die aktuellen Regeln berücksichtigt.

Wenn alles eingestellt wurde, können nun Bilder aus dem Blob Storage ohne Einschränkungen 
in Memoryland von überall auf der Welt angezeigt werden. 
\footnote{Alle Informationen zu CORS für Azure Blob Storage stammen von \cite{MicrosoftCorporationc}}


\subsection{Einrichtung von Azure PostgreSQL DB}

``Azure Database for PostgreSQL'' ist ein Datenbankdienst von Microsoft Azure, 
der auf dem Open-Source-Datenbanksystem PostgreSQL basiert 
\footnote{Weiter Infos zu Azure PostgreSQL: \cite{MicrosoftCorporationf}}. 
Er bietet eine Lösung für Anwendungen, die eine relationale Datenbank benötigen.

Die PostgreSQL Datenbank war eine schlüssige Entscheidung aufgrund der Erfahrung des Backend-Entwicklers.

\begin{table}[h t]
    \centering
    \caption{Vergleich: Flexible Server vs. Single Server}
    \label{tab:azure-postgresql}
    \begin{tabular}{lcc}
    \hline
    \textbf{Feature}                & \textbf{Single Server} & \textbf{Flexible Server} \\ \hline
    \textbf{Betriebssystem}         & Windows                & Linux                    \\
    \textbf{Maximale Speichergrö\ss{}e} & Bis zu 16 TB           & Bis zu 64 TB             \\
    \textbf{Hochverfügbarkeit}      & Nein                   & Ja (Zonenredundant)      \\
    \textbf{Kosten}                 & 1x                     & 2x (Compute + Storage)   \\ \hline
    \end{tabular}
\end{table}

Flexible Server haben gegenüber Single Server Unterschiede. 
Der wichtigste Grund für die Entscheidung war, dass Flexible Server auf Linux basiert, 
während Single Server Windows verwendet. 
\footnote{Warum Linux als Server-Betriebssystem verwendet wurde: \cite{hussain2015survey}}
\footnote{Alle Informationen zu ``Flexible Server vs. Single Server'': \cite{MicrosoftCorporatione}}

Au\ss{}erdem besitzen Flexible Servers eine grö\ss{}ere maximale Speichergrö\ss{}e, zonenredundante Hochverfügbarkeit, 
wodurch Ausfälle besser abgefangen werden können. Obwohl die Kosten etwas höher sein können, 
sind Flexible Servers aufgrund dieser Punkte die bessere Wahl für unser Projekt.

\subsection{Einrichtung von Azure AD B2C}

Die Einrichtung von Azure AD B2C wurde anhand des Beispiels im folgenden MSAL-Beispiel-Projekt
für Angular durchgeführt \cite{MicrosoftCorporationh}. Zusätzlich wurde die folgende
Website verwendet \cite{MicrosoftCorporationg}.
\footnote{Was ist \emph{Azure AD B2C} und warum verwenden wir es: Kapitel \ref{subsection:azure_ad_b2c}}

\subsection{Einrichtung von Azure WebApp}

Die Einrichtung der Azure WebApp wurde mithilfe der folgenden Website erledigt 
\cite{MicrosoftCorporationi}.
\footnote{Was ist \emph{Azure WebApp} und warum verwenden wir es: Kapitel \ref{subsection:azure_web_app}}


\subsection{Einrichtung von Azure Static WebApp}

Die Azure WebApp wurde nicht speziell konfiguriert, es gibt aber einige wichtige Aspekte, 
die vor dem Deployment der Angular Single Page Application beachtet werden müssen.  

Damit die Static WebApp die Routen korrekt erkennt und an die Single Page Application 
weiterleitet, anstatt ``\emph{404 Not Found}''-Fehler auszugeben, muss die folgende 
Konfiguration \ref{lst:staticwebapp-config} in der Datei ``\emph{staticwebapp.config.json}'' hinterlegt werden. 
Diese Datei muss sich im Verzeichnis des Angular-Projekts befinden -- in dieser Diplomarbeit
also im Ordner ``\emph{memoryland}''. \footnote{Infos zu dem Code \ref{lst:staticwebapp-config} gibt es in diesem Video: \cite{MicrosoftCorporationj}}
\footnote{Was ist \emph{Azure Static WebApp} und warum verwenden wir es: Kapitel \ref{subsection:azure_static_web_app}}

\begin{lstlisting}[numbers=left,caption={staticwebapp.config.json},label={lst:staticwebapp-config}]
{
    "navigationFallback": {
        "rewrite": "/",
        "exclude": [
            "*.{png,jpg,jpeg,gif,ico,css,scss,svg,js}"
        ]
    }
}
\end{lstlisting}


\subsection{Einrichtung des Backends}

Die Einrichtung des Backends für die Diplomarbeit erfordert die Konfiguration 
der ``\emph{appsettings.json}'' Datei. In dieser Datei stehen mehrere Einstellungen,
damit Memoryland auf die unterschiedlichen Dienste zugreifen kann. Die Konfiguration 
umfasst drei Hauptbereiche: Authentifizierung, Datenbank und den Azure Blob Storage.

\subsubsection{Authentifizierung}

Für die Authentifizierung wird Azure AD B2C genutzt, um die Benutzerverwaltung zu ermöglichen. 
In der ``\emph{appsettings.json}'' Datei sind die entsprechenden Einstellungen wie in dem folgendem
Beispiel (siehe Listing \ref{lst:appsettings-auth-config}) einzutragen. Hier werden unter anderem die Instance, ClientId und Domain des 
Azure AD B2C Verzeichnisses angegeben. Der SignedOutCallbackPath gibt den Pfad an, der nach 
der Abmeldung des:der Benutzers:in aufgerufen wird, und SignUpSignInPolicyId enthält 
den Namen der entsprechenden Policy. 
\footnote{Alle Informationen zu den Einstellungen der Authentifizierung sind hier zu finden: \cite{MicrosoftCorporationk}}

\begin{lstlisting}[numbers=left,caption={appsettings.json},label={lst:appsettings-auth-config}]
{
    "AzureAdB2C": {
        "Instance": "",
        "ClientId": "",
        "Domain": "",
        "SignedOutCallbackPath": "/signout/<SignUpSignInPolicyId>",
        "SignUpSignInPolicyId": ""
    }
}
\end{lstlisting}

\subsubsection{Datenbank}

Für die Konfiguration der Datenbank sind die entsprechenden Einstellungen, wie in dem folgendem
Beispiel (siehe Listing \ref{lst:appsettings-db-config}), ebenso in der Datei ``\emph{appsettings.json}'' 
einzutragen. Es gibt sowohl eine Einstellung für den Deployment-Datenbankanschluss (Default), 
als auch eine für den lokalen Anschluss (DefaultLocal), der für Test-Zwecke verwendet wird. 
Die UseLocalDb Einstellung gibt an, ob eine lokale Datenbank verwendet werden soll, wobei
\emph{false} bedeutet, dass die Deployment Datenbank verwendet werden soll.
\footnote{Alle Informationen zu den Einstellungen der Azure Datenbank sind hier zu finden: \cite{MicrosoftCorporationl}}

\begin{lstlisting}[numbers=left,caption={appsettings.json},label={lst:appsettings-db-config}]
{
    "UseLocalDb": false,
    "ConnectionStrings": {
        "Default": "",
        "DefaultLocal": ""
    }
}
\end{lstlisting}

\subsubsection{Azure Blob Storage}

Für die Konfiguration des Azure Blob Storage sind die entsprechenden Einstellungen, wie in 
dem folgendem Beispiel (siehe Listing \ref{lst:appsettings-blob-storage-config}), ebenso in der Datei 
``\emph{appsettings.json}'' einzutragen. Diese Einstellung ist notwendig, um die 
Anwendung mit dem Azure Blob Storage zu verbinden und den Zugriff auf Bilder und 
andere Dateien sicherzustellen.
\footnote{Alle Informationen zu den Einstellungen des Azure Blob Storage sind hier zu finden: \cite{MicrosoftCorporationm}}

\begin{lstlisting}[numbers=left,caption={appsettings.json},label={lst:appsettings-blob-storage-config}]
{
    "ConnectionStrings": {
        "BlobStorageDefault": "",
    }
}
\end{lstlisting}


\subsection{Einrichtung des Frontends}

Auch für das Frontend gibt es wichtige Einstellungen, damit die Single Page Application
funktioniert. Hierbei steht die Konfiguration in der Datei ``\emph{environment.ts}''.
Die Konfiguration ist für die Integration der Azure AD B2C Authentifizierung, sowie die 
Verbindung zum Backend und den APIs. Die Datei enthält alle wichtigen Daten für 
die Authentifizierung und die Verbindung zum Backend. 

Im Folgenden ist die Konfiguration beschrieben und in zwei Hauptkapitel unterteilt: 
\emph{environment.ts}-Konfiguration und Deployment.


\subsubsection{\emph{environment.ts}-Konfiguration}

Für die Authentifizierung wird Azure AD B2C verwendet. Die Konfiguration umfasst 
die folgende Definition (siehe Listing \ref{lst:environment-ts}) von Login- und Profil-Bearbeitungs-Flows.
\footnote{Die Informationen zu den Einstellungen stammen alle aus den folgenden Links: \cite{MicrosoftCorporationh} \cite{MicrosoftCorporationg}}

Userflows in Azure AD B2C sind vordefinierte, anpassbare Prozesse für häufige Benutzeraktionen 
wie Anmeldung, Registrierung, Profilbearbeitung oder Passwortzurücksetzung. Sie ermöglichen 
eine schnelle Konfiguration dieser Schritte, sodass Benutzer:innen nach Abschluss Zugriff auf 
die Anwendung erhalten.
\footnote{Die Informationen zu Userflows stammen alle aus dem folgenden Link: \cite{MicrosoftCorporationn}}

\begin{table}[h t]
  \centering
  \caption{Erklärung der Werte von ``\emph{environment.ts}''}
  \label{tab:environment-ts}
  \begin{tabular}{|l|l|}
  \hline
  \textbf{Wert}   & \textbf{Erklärung}                                                  \\ \hline
  signUpSignIn    & Der Name des Sign-Up/Sign-In Flows                                  \\ \hline
  editProfile     & Der Name des Profilbearbeitungs-Flows                               \\ \hline
  authority       & URL der Autorität, die für Authentifizierung genutzt wird           \\ \hline
  authorityDomain & Der Domainname des B2C-Tenants                                      \\ \hline
  clientId        & Die Client-ID der registrierten Anwendung                           \\ \hline
  redirectUri     & URI, wohin nach erfolgreicher Authentifizierung weitergeleitet wird \\ \hline
  scopes          & Liste von Berechtigungen, die die Anwendung benötigt                \\ \hline
  uri             & Die URI des Backend-Servers                                         \\ \hline
  \end{tabular}
\end{table}

\begin{lstlisting}[numbers=left,caption={environment.ts},label={lst:environment-ts}]
const b2cPolicies = {
  names: {
    signUpSignIn: "<flow-name>",
    editProfile: "<flow-name>"
  },
  authorities: {
    signUpSignIn: {
      authority: "https://<tenant-name>.b2clogin.com/
        <tenant-name>.onmicrosoft.com/<flow-name>"
    },
    editProfile: {
      authority: "https://<tenant-name>.b2clogin.com/
        <tenant-name>.onmicrosoft.com/<flow-name>"
    }
  },
  authorityDomain: "<tenant-name>.b2clogin.com"
};
    
export const environment = {
  production: false,
  b2cPolicies: b2cPolicies,
  msalConfig: {
    auth: {
      clientId: "<your-client-id>",
      authority: b2cPolicies.authorities.signUpSignIn.authority,
      knownAuthorities: [b2cPolicies.authorityDomain],
      redirectUri: '/',
    }
  },
  apiConfig: {
    scopes: [
      "https://<tenant-name>.onmicrosoft.com/<app-name>/<scope-name>",
    ],
    uri: "<backend-uri>"
  }
};    
\end{lstlisting}

Diese Konfiguration ermöglicht es, die Anwendung mit den Azure AD B2C 
Flows zu verbinden und eine sichere Kommunikation mit dem Backend sicherzustellen.

\subsubsection{Deployment}

All diese Einstellungen werden auch in GitHub (https://github.com/) benötigt.
Mithilfe von GitHub Actions \footnote{Alle Informationen zu GitHub stammen von hier: \cite{GitHuba}}
wird unser Frontend bei einem Push oder einer Pull-Request auf den Branch Main deployed.
Dafür müssen jedoch die folgenden Variablen \ref{tab:github-secrets} mittels GitHub Secrets im GitHub Repo eingetragen werden.
\footnote{Alle Informationen zu GitHub Secrets stammen von hier: \cite{GitHub}}

\begin{table}[h t]
    \centering
    \caption{GitHub Secrets für das Deployment des Frontends}
    \label{tab:github-secrets}
    \begin{tabular}{|l|l|}
    \hline
    \textbf{GitHub Secret}              & \textbf{Beschreibung} \\ \hline
    \texttt{B2C\_SIGNUP\_SIGNIN\_FLOW}   & Name des Sign-Up/Sign-In Flows \\ \hline
    \texttt{B2C\_EDIT\_PROFILE\_FLOW}    & Name des Profilbearbeitungs-Flows \\ \hline
    \texttt{B2C\_SIGNUP\_SIGNIN\_AUTHORITY} & Autoritäts-URL für den Sign-Up/Sign-In Flow \\ \hline
    \texttt{B2C\_EDIT\_PROFILE\_AUTHORITY}  & Autoritäts-URL für den Profilbearbeitungs-Flow \\ \hline
    \texttt{B2C\_AUTHORITY\_DOMAIN}      & Domain des B2C-Tenants \\ \hline
    \texttt{CLIENT\_ID}                  & Client-ID der registrierten Anwendung \\ \hline
    \texttt{SCOPE\_READ}                 & API-Berechtigung für Lesezugriff \\ \hline
    \texttt{SCOPE\_WRITE}                & API-Berechtigung für Schreibzugriff \\ \hline
    \texttt{BACKEND\_URI}                & URI des Backend-Servers \\ \hline
    \end{tabular}
    \caption{Verwendete GitHub Secrets für die Konfiguration}
\end{table}

