\setauthor{Arwed Schnalzenberger}

\section{Einrichtung}
\input{sections/backend-getting-started.tex}

\section{Technologien}
\subsection{.NET mit C\#}

C\# ist eine objektorientierte Programmiersprache, die speziell für die Entwicklung von 
Anwendungen auf der .NET-Plattform optimiert wurde. Als eine der fünf beliebtesten 
Programmiersprachen auf GitHub ist sie vollständig ``open-source''.

C\# greif die Konzepte aus JavaScript, Java und C++ auf, was sie für Leute mit Kenntnissen
in diesen Bereichen leichter zu lernen macht. Zu den zentralen Sprachmerkmalen gehören Generics, 
Pattern Matching, asynchrone Programmierung und Records. Diese Funktionen ermöglichen eine 
typsichere und strukturierte Entwicklung.

Für die Arbeit mit C\# stehen verschiedene Entwicklungsumgebungen und Werkzeuge zur Verfügung. 
Visual Studio bietet eine integrierte Entwicklungsumgebung mit umfangreichen Funktionen. 
Visual Studio Code stellt eine leichtgewichtige Alternative mit Erweiterungsmöglichkeiten 
dar. Zudem ermöglichen Kommandozeilen-Tools (CLI-Tools) eine flexible Nutzung. JetBrains Rider 
ist eine weitere IDE, die speziell für .NET optimiert wurde und in diesem Projekt genutzt 
wird. \ref{subsection:jetbrains-rider}

In dieser Diplomarbeit wurde C\# für das REST-Backend verwendet. Die zentrale Datenverwaltung 
erfolgte über Entity Framework Core mit einer PostgreSQL-Datenbank. Die Authentifizierung 
wurde durch die Anbindung an Azure AD B2C mittels MSAL realisiert. REST-APIs wurden implementiert, 
um die Kommunikation mit Unity und dem Frontend zu ermöglichen. Zusätzlich wurde Azure Blob 
Storage zur Speicherung und Verwaltung von Bilddateien integriert.

Die Wahl von C\# basiert auf zwei Faktoren. Die Sprache bietet eine direkte Integration mit 
Microsoft-Diensten, einschließlich der genutzten Azure-Dienste. Zudem wird C\# am meisten im 
Programmier-Zweig der HTL-Leonding unterrichtet, sodass bereits Vorkenntnisse vorhanden 
waren, die in der Diplomarbeit genutzt werden konnten.
\footnote{Alle Informationen zu C\# stammen von: \cite{MicrosoftCorporationo}}

\subsection{Entity Framework Core}

Entity Framework Core ist eine open-source Version des Entity Frameworks. Es dient 
als objektrelationaler Mapper (O\slash RM) für .NET-Anwendungen. Entity Framework Core ermöglicht 
die Arbeit mit relationalen Datenbanken unter Verwendung von .NET-Objekten und reduziert 
dabei den Aufwand für manuelle Datenzugriffslogik.

Der Datenzugriff erfolgt über ein Modell, das sich aus Entitätsklassen und einem 
Kontextobjekt zusammensetzt, welches bei uns ``ApplicationDbContext'' heißt. Dieses Kontextobjekt 
stellt eine Verbindung zur Datenbank her und verwaltet Datenbankabfragen sowie das Speichern 
von Daten. Entity Framework Core unterstützt verschiedene Ansätze zur Modellerstellung, 
darunter das Generieren eines Modells aus einer bestehenden Datenbank oder das manuelle 
Erstellen eines Modells, das anschließend mithilfe von Entity Framework Migrations in 
eine Datenbank überführt werden kann. In dieser Diplomarbeit wurde das Modell erstellt und dann
in eine Datenbank überführt.

Die Abfrage von Daten erfolgt über Language Integrated Query (LINQ). Änderungen an den Daten, 
wie das Erstellen, Modifizieren oder Löschen von Entitäten, werden über den DBContext 
vorgenommen und persistiert. Dabei stehen verschiedene Methoden zur Verfügung, 
um Daten effizient zu verwalten.
\footnote{Alle Informationen zu LINQ stammen von: \cite{MicrosoftCorporationq}}

Bei der Nutzung von Entity Framework Core sind verschiedene Aspekte zu beachten. Ein 
grundlegendes Verständnis relationaler Datenbanken ist erforderlich, um komplexe 
Datenbankoperationen effizient zu gestalten. Die Performance kann durch gezielte 
Optimierungen, wie den gezielten Einsatz von Indizes oder die Vermeidung von nicht 
skalierbaren Abfragen, verbessert werden.
\footnote{Alle Informationen zu Entity Framework Core stammen von: \cite{MicrosoftCorporationp}}


\subsection{MSAL}

MSAL (Microsoft Authentication Library) ermöglicht Entwicklern das Abrufen von Sicherheitstoken 
von der Microsoft-Identity-Platform, um Benutzer zu authentifizieren und auf gesicherte Web-APIs 
zuzugreifen. Sie kann verwendet werden, um sicheren Zugriff auf Microsoft Graph, andere 
Microsoft-APIs, Drittanbieter-Web-APIs oder eigene Web-APIs bereitzustellen. MSAL unterstützt 
viele verschiedene Anwendungsarchitekturen und Plattformen, darunter .NET, JavaScript, Java, 
Python, Android und iOS.

Für diese Diplomarbeit wurde MSAL verwendet, um eine benutzerdefinierte Authentifizierungslösung 
für das Angular-Frontend zu implementieren, die eine sichere Anmeldung und den Zugriff auf 
Web-APIs ermöglicht. Besonders hilfreich war MSAL bei der Anbindung an Azure AD B2C, da es 
eine einfache Handhabung der Authentifizierung und Tokenverwaltung ermöglichte.

Die Wahl von MSAL basiert auf der umfassenden Unterstützung durch Microsoft und der 
Integration in Azure-Dienste. Zudem wird MSAL in vielen verschiedenen Plattformen und 
Architekturen unterstützt, was es zu einer flexiblen Lösung für verschiedene Anwendungsfälle 
macht.
\footnote{Alle Informationen zu MSAL stammen von: \cite{MicrosoftCorporationmsal}}


\subsection{REST}

\subsection{Postgres-DB}
\label{subsection:postgres_db}

\subsection{Cloud-Dienste mit Azure}

\subsubsection{Azure Blob Storage}
\label{subsection:azure_blob_storage}

\subsubsection{Azure AD B2C}
\label{subsection:azure_ad_b2c}

https://learn.microsoft.com/en-us/azure/active-directory-b2c/overview

https://learn.microsoft.com/en-us/azure/active-directory-b2c/

\subsubsection{Azure WebApp}
\label{subsection:azure_web_app}

\subsubsection{Azure Static WebApp}
\label{subsection:azure_static_web_app}

\subsection{JetBrains Rider}
\label{subsection:jetbrains-rider}

\subsection{CI/CD mit GitHub Actions}


\section{API}
In Memoryland wird eine Web-API eingesetzt, um die Verwaltung von Bildern, Alben
und anderen Daten im Backend zu steuern. Die API übernimmt dabei die Speicherung, 
Strukturierung und den Zugriff auf diese Daten, während das Frontend über die API 
mit dem Backend kommuniziert.

\subsection{Arbeiten mit C\#-Controllern}

Die API wurde mittels einer C\# WebAPI mit Controllern entwickelt. Diese auf dem .NET-Framework
basierende Technologie ermöglicht es uns RESTful-Endpunkte zu erstellen, über die das Frontend
mit dem Backend kommunizieren kann. Controller helfen uns dabei die unterschiedlichen Funktionen
des Backends in eigene Klassen zu unterteilen, was eine klarere Struktur ermöglicht.
\footnote{Alle Infos zu C\# WebAPIs mit Controllern stammen von: \cite{MicrosoftCorporationaa}}

Um Code-Verdoppelung zu vermeiden, wurde im Memoryland-Backend eine Basisklasse 
für API-Controller implementiert. Von der Klasse ``\emph{ApiControllerBase}'' (siehe Listing \ref{lst:api-controller-base}) 
erben alle anderen Controller, wodurch eine einheitliche Struktur sichergestellt und die Wartung 
erleichtert wird.

Die Base-Controller Klasse ist mit den zwei Attributen ``[ApiController]'' und 
``[Route(``api/[controller]'')]'' versehen. Mit dem ``[ApiController]''-Attribut 
erkennt das .Net-Framework die Klasse als API-Controller. Das 
``[Route(``api/[controller]'')]''-Attribut hingegen legt die Basisroute für alle Endpunkte 
der Controller fest, die von der ApiControllerBase erben. 
Der Platzhalter \emph{controller} wird durch den Namen des jeweiligen Controllers ersetzt. 
Beispielsweise erhält die UploadController-Klasse dadurch die Basisroute ``api/Upload''.

\begin{lstlisting}[numbers=left,caption={APIControllerBase.cs},label={lst:api-controller-base}]
using Microsoft.AspNetCore.Mvc;

namespace WebApi.Controllers;
    
[ApiController]
[Route("api/[controller]")]
public class ApiControllerBase : ControllerBase { }    
\end{lstlisting}

\subsubsection{Probleme bei der Implementierung}

Bei der Implementierung dieser Struktur kam es zu dem Problem, dass in ein paar Methodenrouten
ein führende Schrägstrich ``\slash'' verwendet wurde. Dies überschreibt jedoch die definierte
Basisroute des Controllers.

Beispielsweise führt die Angabe ``[HttpPost(``/picture'')]'' dazu, dass die Route nicht 
``api/Upload/picture'' lautet, sondern nur ``/picture''. Daher sollte in der Methodenroute 
kein führender Schrägstrich ``\slash'' verwendet werden.

Dies führte zu Problemen auch mit Swagger und OpenAPI, da diese Middleware versuchte, 
die Methode dem entsprechenden Controller zuzuordnen. Da der Controller eine Basisroute 
definiert hatte, jedoch die tatsächliche Route aufgrund des beschriebenen Problems nicht 
in diese Struktur eingebunden war, warf die Middleware den Error. Dies führte dazu, 
dass Swagger und OpenAPI die Methode nicht korrekt dokumentieren oder aufrufen konnten.

\subsection{Endpunkte in C\# Controllern}

Die Methode MyPostEndpointMethod (siehe Listing \ref{lst:endpoint-example}) stellt einen HTTP-POST-Endpunkt in einer Web-API bereit und
wird als Beispiel für die folgende Erklärung der Implementierung von Endpunkten in C\# Controllern
verwendet. Die Methode ist mit mehreren Attributen versehen, die Authentifizierung, 
Autorisierung und die API Route steuern. Das ``[Authorize]''-Attribut stellt sicher, 
dass nur authentifizierte Benutzer:innen Zugriff auf die Methode erhalten, wobei es dafür
die in der Datei ``Program.cs'' definierte Technologie verwendet. Zusätzlich überprüft 
``[RequiredScope(``backend.write'')]'', ob der Benutzer:innen über die notwendigen Berechtigungen 
verfügt. Diese Überprüfung basiert auf OAuth 2.0 und OpenID Connect und stellt sicher, 
dass nur Benutzer:innen mit den nötigen Rechten (Scopes) bestimmte Operationen durchführen können. 
(siehe Kapitel \ref{subsection:azure_ad_b2c})

Mit dem ``[Route(...)]''-Attribut wird die Route der Methode definiert. Hierbei können in
geschwungenen Klammern ``\{ \}'' URL-Parameter definiert werden. Es können aber auch 
Objekte im Body des HTTP-Aufrufs mitgegeben werden. Dafür benötigt man das 
Attribut ``[FromBody]'' vor dem Parameter der Funktion setzen. Das JSON-Objekt wird dann
automatisch in die Objekt-Instanz deserialisiert und kann dann normal genutzt werden. 

Schlussendlich gibt die Methode dann ein Results-Objekt zurück, welches mehrere 
mögliche Antworttypen umfasst. Diese Antworttypen beinhalten die HTTP-Status-Werte.
Bei erfolgreicher Erstellung eines neuen Objekts wird also ein Created-Ergebnis zurückgegeben. 
Dies entspricht dem Status-Code 201. Falls die Anfrage fehlerhafte oder unvollständige Daten 
enthält, wird ein BadRequest<string> zurückgeliefert, welches auch zusätzliche Informationen
im ``string''-Format enthalten kann. Ist die Verarbeitung erfolgreich, aber kein neues Objekt 
erforderlich, wird stattdessen ein Ok<ReturnObj> zurückgegeben, welches wieder bestimmte
Typen enthalten kann, wie eine Liste an Bildern.
\footnote{Alle Informationen zur Implementierung von Endpunkten in C\# Controllern stammen von: \cite{MicrosoftCorporationaa} \cite{MicrosoftCorporationab}}

\begin{lstlisting}[numbers=left,caption={Beispiel eines Endpunkts},label={lst:endpoint-example}]
[HttpPost]
[Authorize]
[Route("mypath/{objId:long}")]
[RequiredScope("backend.write")]
public async Task<Results<Created, BadRequest<string>, Ok<ReturnObj>>> 
    MyPostEndpointMethod(
    long objId, 
    [FromBody] ObjDto postObjDto)
{
    ...
}
\end{lstlisting}

\subsection{API-Dokumentation mit Swagger und OpenApi}

Für eine übersichtlichere Dokumentation der API, wurde Swagger und OpenAPI eingesetzt.
OpenAPI ist eine Spezifikation zur Beschreibung und Dokumentationvon REST-APIs. Eine
OpenAPI-Datei enthält alle Informationen zu der API, einschließlich der verfügbaren 
Endpunkte, sowie der Ein- und Ausgaben. Diese JSON Spezifikation wird in diesem Projekt
automatisch erstellt.
\footnote{Ein Beispiel einer Swagger-UI ist im Bild \ref{fig:swagger-ui-example} zu sehen.}

Durch den Einsatz von Swagger und OpenAPI wurde die Nutzung der API erleichtert, 
da die Frontend-Entwicklerin die verfügbaren Endpunkte und deren Funktionalitäten direkt 
einsehen und testen konnten. 
\footnote{Alle Informationen zu Swagger und OpenAPI stammen von: \cite{SmartBearSoftware}}

\subsection{Erklärung der Controller und ihrer Aufgaben}

Die vier Controller in der API des Memoryland-Bakcends erfüllen jeweils spezifische 
Aufgaben im Zusammenhang mit der Verwaltung von Fotos, Alben, Uploads und deren Transaktionen.

\subsubsection{Album-Controller}

Der AlbumController ist zuständig für die Verwaltung von Fotoalben. Dieser Controller 
ermöglicht das Erstellen, Abrufen und Löschen von Alben, sowie das Zuweisen 
von Fotos zu Alben. (siehe Tabelle \ref{tab:album-controller})

\begin{table}[h t]
    \centering
    \caption{Album Controller Endpunkte}
    \label{tab:album-controller}
    \begin{tabular}{|l|p{5cm}|l|p{5cm}|}
    \hline
    \textbf{Methode} & \textbf{Pfad} & \textbf{Authorized} & \textbf{Beschreibung} \\ \hline
    GET & /api/PhotoAlbum\break{/\{albumId\}} & Ja & Ruft ein Fotoalbum mit der angegebenen ID ab. \\ \hline
    GET & /api/PhotoAlbum & Ja & Ruft eine Liste aller Fotoalben ab. \\ \hline
    PUT & /api/PhotoAlbum & Ja & Bearbeitet den Namen eines Fotoalbums. \\ \hline
    POST & /api/PhotoAlbum\break{/\{albumName\}} & Ja & Erstellt ein neues Fotoalbum mit dem angegebenen Namen. \\ \hline
    DELETE & /api/PhotoAlbum\break{/\{photoAlbumId\}} & Ja & Löscht ein Fotoalbum mit der angegebenen ID. \\ \hline
    \end{tabular}
\end{table}


\subsubsection{Memoryland-Controller}
\label{backend-memoryland-controller}

Der MemorylandController ist für die Verwaltung und Interaktion mit Memorylands 
verantwortlich. Dabei werden Memorylands, die Positionierung der Bilder in Memorylands
und Tokens für den Zugriff auf Memorylands verwaltet. (siehe Tabelle \ref{tab:memoryland-controller})

Worum es sich bei konfigurationen von Memorylands handelt, ist im Kapitel 
\ref{sec:memoryland-config} beschrieben.

\begin{table}[h t]
    \centering
    \caption{Memoryland Controller Endpunkte}
    \label{tab:memoryland-controller}
    \begin{tabular}{|l|p{5cm}|l|p{5cm}|}
    \hline
    \textbf{Methode} & \textbf{Pfad} & \textbf{Authorized} & \textbf{Beschreibung} \\ \hline
    GET & /api/Memoryland/all & Ja & Gibt eine Liste aller Memorylands zurück. \\ \hline
    GET & /api/Memoryland & \textbf{Nein} & Gibt eine Memoryland mit all dessen Bildern zurück. Dafür wird ein Token benötigt \\ \hline
    PUT & /api/Memoryland & Ja & Bearbeitet den Namen eines Memorylands. \\ \hline
    GET & /api/Memoryland/\{id\}\break/configuration & Ja & Gibt die Konfiguration für das angegebene Memoryland zurück. \\ \hline
    GET & /api/Memoryland/types & \textbf{Nein} & Gibt die verfügbaren Typen für Memorylands zurück. \\ \hline
    GET & /api/Memoryland/\{id\}\break/token & Ja & Gibt das Token für das angegebene Memoryland zurück. \\ \hline
    POST & /api/Memoryland/\{id\}\break/token & Ja & Erstellt ein neues Token für das angegebene Memoryland. \\ \hline
    POST & /api/Memoryland\break/\{memorylandName\}\break/{memorylandTypeId} & Ja & Erstellt ein Memoryland mit dem angegebenen Namen und Typ. \\ \hline
    POST & /api/Memoryland\break/\{memorylandId\} & Ja & Erstellt eine Konfiguration für das Memoryland. \\ \hline
    DELETE & /api/Memoryland\break/\{memorylandId\} & Ja & Löscht das angegebene Memoryland. \\ \hline
    DELETE & /api/Memoryland/config\break/\{id\} & Ja & Löscht die angegebene Konfiguration eines Memorylands. \\ \hline
    \end{tabular}
\end{table}


\subsubsection{Foto-Controller}

Der FotoController ist für die Verwaltung von Fotos verantwortlich. Dies umfasst das 
Abrufen, Erstellen und Löschen von Fotos, sowie das Verwalten der Fotodaten, die in 
Alben gespeichert sind. (siehe Tabelle \ref{tab:foto-controller})

\begin{table}[h t]
    \centering
    \caption{Foto Controller Endpunkte}
    \label{tab:foto-controller}
    \begin{tabular}{|l|p{5cm}|l|p{5cm}|}
    \hline
    \textbf{Methode} & \textbf{Pfad} & \textbf{Authorized} & \textbf{Beschreibung} \\ \hline
    GET & /api/Photo/\{albumId\}\break/\{photoName\} & Ja & Ruft ein Foto aus einem Album ab. \\ \hline
    DELETE & /api/Photo/\{photoId\} & Ja & Löscht ein Foto mit der angegebenen ID. \\ \hline
    PUT & /api/Photo & Ja & Bearbeitet den Namen eines Fotos. \\ \hline
    \end{tabular}
\end{table}


\subsubsection{Upload-Controller}

Der UploadController ist für das Hochladen von Fotos und das Öffnen von 
Transaktionen zuständig. (siehe Tabelle \ref{tab:upload-controller})

\begin{table}[h t]
    \centering
    \caption{Upload Controller Endpunkte}
    \label{tab:upload-controller}
    \begin{tabular}{|l|p{5cm}|l|p{5cm}|}
    \hline
    \textbf{Methode} & \textbf{Pfad} & \textbf{Authorized} & \textbf{Beschreibung} \\ \hline
    GET & /api/Upload/transaction & Ja & Ruft alle Upload-Transaktionen ab. \\ \hline
    POST & /api/Upload/transaction & Ja & Erstellt eine neue Upload-Transaktion. \\ \hline
    POST & /api/Upload/photo & Ja & Lädt ein Foto hoch. \\ \hline
    DELETE & /api/Upload/transaction\break/\{transactionId\} & Ja & Löscht eine Upload-Transaktion mit der angegebenen ID. \\ \hline
    \end{tabular}
\end{table}




\section{Integration von Azure Blob Storage}

Die Speicherung und Verwaltung von Bildern in der Memoryland Diplomarbeit erfolgt 
mithilfe einer Kombination aus einer relationalen PostgreSQL Datenbank und einem 
Azure Blob Storage Cloud-Speicher. Während PostgreSQL die Metadaten der Bilder 
speichert, wie in welchem Album sich welches Bild befindet, werden die 
eigentlichen Bilddateien in Azure Blob Storage abgelegt.

Azure Blob Storage bietet also eine skalierbare Lösung für die Speicherung 
von Bilddateien. Durch die Kombination mit PostgreSQL zur Speicherung von Strukturen 
ist eine effizientere Verwaltung und ein einfacherer Zugriff möglich. Der Zugriff 
über SAS-Tokens ermöglicht es au\ss{}erdem, Bilder sicher und kontrolliert bereitzustellen, 
ohne direkte Anmeldeinformationen preiszugeben.
\footnote{Alle Informationen zu SAS-Tokens stammen von hier: \cite{MicrosoftCorporationa}}
\footnote{Weiter Informationen zu Azure Blob Storage sind im Kapitel \ref{subsection:azure-blob-storage-getting-started}}
\footnote{Weiter Informationen zu der PostgreSQL Datenbank-Struktur sind im Kapitel \ref{chapter:datamodel}}

\section{Blob-Storage Bilder-Verwaltung}

Im Blob Storage werden für jede:n Benutzer:in ein separater Blob-Container erstellt. 
Die Bilder werden im Blob-Storage durch die in der PostgreSQL festgelegten eindeutigen
Identifikationsnummer gespeichert, sodass es zu keinen Problemen kommen kann, wenn zwei
Foto in unterschiedlichen Alben gleich hei\ss{}en.

\subsection{Azure Blob-Storage Datenhaltung}
\label{subsection:azure_blob_storage_datamodel}

Die Speicherung erfolgt also in einer flachen Struktur innerhalb des 
jeweiligen Benutzer:innen-Containers:

/<userId>/\\
 - <photoId>.jpg\\
 - <photoId>.png\\
 - ...\\

Die Speicherung eines Bildes erfolgt über den BlobStoragePhotoService. 
Dabei wird zunächst überprüft, ob der Container für den:die Benutzer:in existiert und falls dies 
nicht der Fall ist, wird er erstellt. Anschlie\ss{}end wird das Bild hochgeladen.
(siehe Listing \ref{lst:blob-storage-photo-upload})
\footnote{Alle Informationen zu dem Upload von Bildern auf Azure Blob Storage stammen von hier: \cite{MicrosoftCorporationac}}

\begin{lstlisting}[numbers=left,caption={Foto Upload auf den Blob-Storage},label={lst:blob-storage-photo-upload}]
public async Task UploadPhoto(
    long userId, 
    long photoId, 
    string photoName, 
    byte[] photoBytes)
{
    var containerClient = BlobSvcClient.GetBlobContainerClient(PadLong(userId));
    await containerClient.CreateIfNotExistsAsync();
    
    var blobClient = containerClient.GetBlobClient(PadLong(photoId) + GetExtension(photoName));
    
    using var stream = new MemoryStream(photoBytes);
    await blobClient.UploadAsync(stream, overwrite: true);
    
    await blobClient.SetHttpHeadersAsync(new BlobHttpHeaders
    {
        ContentDisposition = "inline",
        ContentType = GetMimeType(photoName)
    });
}    
\end{lstlisting}

\subsubsection{Probleme bei der Implementierung}

Der HTTP-Header ``\emph{Content-Disposition: inline}'' wurde spezifisch gesetzt,
da es davor das Problem, dass Bilder automatisch im Browser heruntergeladen wurden.
``\emph{inline}'' gibt an, dass eine Datei direkt im Browser angezeigt werden kann.
\footnote{Alle Informationen}

\subsection{Rotieren der Bilder bei der Speicherung}

Beim Laden von PNG-Dateien in Unity trat das Problem auf, dass die Bildorientierung 
nicht immer korrekt war. Dies lag daran, dass die in den EXIF-Daten gespeicherte 
Rotationsinformation nicht durchgehend berücksichtigt wurde. Um die Bilder 
korrekt darzustellen, wurde die folgende Methode implementiert, die die 
EXIF-Rotationsdaten ausliest und das Bild entsprechend anpasst.

Die Methode RotateImage lädt das Bild, wendet eine mögliche Rotation 
basierend auf den EXIF-Daten an und speichert es anschlie\ss{}end im korrekten Format. 
(Siehe Listing \ref{lst:method-rotate-image})

\begin{lstlisting}[numbers=left,caption={Methode RotateImage},label={lst:method-rotate-image}]
private static byte[] RotateImage(byte[] imageBytes)
{
    using var inputStream = new MemoryStream(imageBytes);
    var options = new DecoderOptions();
    using var image = Image.Load(options, inputStream);
        
    ApplyExifRotation(image);
    
    using var outputStream = new MemoryStream();
    image.Save(outputStream, new SixLabors.ImageSharp.Formats.Jpeg.JpegEncoder());
    return outputStream.ToArray();
}    
\end{lstlisting}

Die Methode ApplyExifRotation überprüft, ob das Bild eine EXIF-Rotationsinformation enthält, 
und wendet die entsprechende Transformation an. 
(Siehe Listing \ref{lst:method-apply-exif-rotation})

\begin{lstlisting}[numbers=left,caption={Methode ApplyExifRotation},label={lst:method-apply-exif-rotation}]
    private static void ApplyExifRotation(Image image)
    {
        if (image.Metadata.ExifProfile == null) return; 
        // Keine EXIF-Daten vorhanden
                
        image.Metadata.ExifProfile.TryGetValue(
            ExifTag.Orientation, 
            out var orientation);
            
        if (orientation == null) return; 
        // Keine Orientierungsinformationen verfügbar
            
        switch (orientation.Value)
        \{
            case 2: // Horizontal spiegeln
                image.Mutate(x => x.Flip(FlipMode.Horizontal));
                break;
            case 3: // 180 Grad drehen
                image.Mutate(x => x.Rotate(RotateMode.Rotate180));
                break;
            ...
        \}
            
        // Die Rotationsinformation wird nach der Korrektur entfernt, 
        // um doppelte Anpassungen zu vermeiden
        image.Metadata.ExifProfile.SetValue(ExifTag.Orientation, (ushort)1);
    }    
\end{lstlisting}

\subsection{SAS-Token Generierung}
\label{subsection:sas-token-generation}

Die Methode ``CreateUserDelegationSasBlob'' (Siehe Listing \ref{lst:sas-token-generator-service}) 
erstellt ein Shared Access Signature (SAS)-Token für einen Azure Blob. Dieses Token ermöglicht 
es, für eine bestimmte Zeit auf das Blob zuzugreifen, ohne die primären Zugangsdaten des 
Azure Storage-Kontos preiszugeben. Die Methode erwartet zwei Parameter. Einen ``BlobClient'', 
der auf das gewünschte Blob verweist, und einen ``accessKey'', der für die Signierung des 
SAS-Tokens verwendet wird.

Zunächst wird die Gültigkeit des SAS-Tokens festgelegt. Der Startzeitpunkt wird auf die 
aktuelle UTC-Zeit gesetzt, und die Endzeit wird um 4 Stunden verlängert. Anschlie\ss{}end wird 
ein ``BlobSasBuilder''-Objekt erstellt, dass dann schlussendlich für die Erstellung des
SAS Tokens zuständig ist. Hierbei wird der ``BlobContainerName'' und der ``BlobName'' 
vom ``BlobClient'' übernommen, um das Blob eindeutig zu identifizieren, zu welchem das 
SAS-Token generiert werden soll. Zudem wird die Ressource auf ``b'' gesetzt, was aussagt, 
dass es sich um einen SAS-Token für einen Blob handelt, und nicht um einen gesamten Container. 
Die Berechtigungen für das Token werden so gesetzt, dass es Lese- und Schreibzugriff 
auf das Blob gewährt.

Ein weiteres Objekt, ``StorageSharedKeyCredential'', wird mit dem ``accountName'' und dem 
``accessKey'' des Azure Storage-Kontos erzeugt. Dieses Credential wird genutzt, um das 
SAS-Token zu signieren, sodass es nur für das spezifische Konto gültig ist.

Mit einem ``BlobUriBuilder'' wird schlie\ss{}lich die vollständige URL erzeugt, die das Blob 
und das SAS-Token enthält. Das Token wird als URL-Parameter angehängt, wodurch der Zugriff 
auf das Blob mit den festgelegten Berechtigungen möglich ist.

Am Ende gibt die Methode die vollständige URL zurück, die auf das Blob verweist und das 
SAS-Token enthält. Diese URL kann dann verwendet werden, um sicher auf das Blob zuzugreifen, 
ohne dass die Zugangsdaten des Azure Storage-Kontos benötigt werden.

\begin{lstlisting}[numbers=left,caption={SasTokenGeneratorService},label={lst:sas-token-generator-service}]
    private static double TokenLifeTimeInHours => 4;

    public static Uri CreateUserDelegationSasBlob(
        BlobClient blobClient,
        string accessKey)
    {
        var tokenStartTime = DateTimeOffset.UtcNow;
        var tokenEndTime = tokenStartTime
            .AddHours(TokenLifeTimeInHours);
        
        var sasBuilder = new BlobSasBuilder()
        {
            BlobContainerName = blobClient.BlobContainerName,
            BlobName = blobClient.Name,
            Resource = "b", // Blob resource
            StartsOn = tokenStartTime,
            ExpiresOn = tokenEndTime
        };

        sasBuilder.SetPermissions(
            BlobSasPermissions.Read | BlobSasPermissions.Write);

        var blobSvcClient = blobClient
            .GetParentBlobContainerClient()
            .GetParentBlobServiceClient();
        
        var storageSharedKeyCredential = new StorageSharedKeyCredential(
            blobSvcClient.AccountName,
            accessKey);

        var uriBuilder = new BlobUriBuilder(blobClient.Uri)
        {
            Sas = sasBuilder.ToSasQueryParameters(storageSharedKeyCredential)
        };

        return uriBuilder.ToUri();
    }
\end{lstlisting}

\section{Uploads}

\subsection{Uploads von einzelnen Fotos}

Die Methode ``UploadPhoto'' (Siehe Listing \ref{lst:method-upload-photo}) ist eine Implementierung 
eines HTTP-POST-Endpoints, der es einem:r Benutzer:in ermöglicht, ein Foto in ein bestimmtes 
Album hochzuladen. Der Endpoint ist mit den Attributen ``[Authorize]'' und 
``[RequiredScope(``backend.write'')]'' versehen, um sicherzustellen, dass nur authentifizierte 
Benutzer:innen mit der entsprechenden Berechtigung auf diese Route zugreifen können.

Zunächst wird der:die Benutzer:in mit der Methode ``CheckIfUserAuthenticated'' (Siehe Listing 
\ref{lst:method-check-user-auth}) überprüft, um sicherzustellen, dass der:die Benutzer:in 
authentifiziert ist und keine Fehler bei der Authentifizierung aufgetreten sind. 
Falls keine Email in den Claims angegeben ist, wird Unauthorized zurückgegeben. 
Zusätzlich erstellt die Methode einen:r Benutzer:in in unserer Datenbank, falls
der:die Benutzer:in noch nicht existiert.

Anschlie\ss{}end wird überprüft, ob das angegebene Fotoalbum existiert. Wenn das 
Album nicht existiert oder der:die Benutzer:in nicht der Eigentümer des Albums ist, 
wird eine ``BadRequest''-Antwort mit einer entsprechenden Fehlermeldung zurückgegeben.

Es folgt eine Validierung des Foto-Uploads. Zunächst wird geprüft, ob eine Bilddatei 
erfolgreich übermittelt wurde. Falls das Bild leer ist, wird ebenfalls eine 
``BadRequest''-Antwort gesendet. Dann wird geprüft, ob der Dateiname des Bildes 
angegeben ist, ob er innerhalb der zulässigen Länge (zwischen 3 und 63 Zeichen) liegt 
und ob er keine ungültigen Zeichen enthält. Wird einer dieser Prüfungen nicht bestanden, 
wird eine entsprechende Fehlermeldung zurückgegeben.

Darüber hinaus wird überprüft, ob bereits ein Foto mit dem gleichen Dateinamen im 
angegebenen Album existiert. Falls dies der Fall ist, wird ebenfalls eine 
``BadRequest''-Antwort mit der Info zurückgegeben, dass der Dateiname bereits vergeben ist.

Wenn alle Validierungen erfolgreich sind, werden zunächst die Metadaten des Fotos in der
PostgreSQL Datenbank gespeichert. Danach wird das Foto in ein Byte-Array umgewandelt und auf
dem Blob Storage gespeichert.

Abschlie\ss{}end gibt die Methode eine ``Created''-Antwort zurück, was anzeigt, dass das Foto 
erfolgreich hochgeladen wurde.


\begin{lstlisting}[numbers=left,caption={Methode UplaodPhoto},label={lst:method-upload-photo}]
 
    [HttpPost]
    [Route("photo")]
    [Authorize]
    [RequiredScope("backend.write")]
    public async Task<Results<Created, BadRequest<string>, UnauthorizedHttpResult>> 
    UploadPhoto([FromForm] PostPhotoDto<IFormFile> photoDto)
    {
        var user = await UserSvc.CheckIfUserAuthenticated(User.Claims, true);
        
        if (user == null) 
            return TypedResults.Unauthorized();
        
        if (!Context.PhotoAlbums.Any(pa => 
                pa.Id == photoDto.PhotoAlbumId &&
                pa.UserId == user.Id))
            return TypedResults.BadRequest("The photo album doesn't exist.");
        
        if (photoDto.Photo.Length == 0)
            return TypedResults.BadRequest("No image file provided.");
        
        if (string.IsNullOrWhiteSpace(photoDto.FileName))
            return TypedResults.BadRequest("FileName name is required");
        
        if (photoDto.FileName.Length < 3 || photoDto.FileName.Length > 63)
            return TypedResults
                .BadRequest("A FileName name can't be longer than 
                    63 characters or shorter than 3");
        
        if (ContainerNameRegex.IsMatch(photoDto.FileName))
            return TypedResults.BadRequest("FileName name contains invalid characters");
        
        var dbPhoto = Context.Photos
            .Include(p => p.PhotoAlbum)
            .FirstOrDefault(p =>
                p.Name == photoDto.FileName &&
                p.PhotoAlbumId == photoDto.PhotoAlbumId &&
                p.PhotoAlbum.UserId == user.Id);
        
        if (dbPhoto != null) 
            return TypedResults.BadRequest("FileName name already exists");
        
        byte[] photoData;
        using (var memoryStream = new MemoryStream())
        {
            await photoDto.Photo.CopyToAsync(memoryStream);
            photoData = memoryStream.ToArray();
        }
        
        var photo = new Photo
        {
            Name = photoDto.FileName,
            PhotoAlbumId = photoDto.PhotoAlbumId
        };
        
        await Context.AddAsync(photo);
        await Context.SaveChangesAsync();
        
        var album = Context.PhotoAlbums
            .Include(photoAlbum => photoAlbum.User)
            .FirstOrDefault(pa => pa.Id == photo.PhotoAlbumId);
        
        await PhotoSvc.UploadPhoto(
            album!.User.Id,
            photo.Id,
            photo.Name,
            photoData);
        
        return TypedResults.Created();
    }
\end{lstlisting}

\begin{lstlisting}[numbers=left,caption={Methode CheckIfUserAuthenticated},label={lst:method-check-user-auth}]
    public async Task<User?> CheckIfUserAuthenticated(
        IEnumerable<Claim> claims, 
        bool createUserIfNotExist = false)
    {
        var claimList = claims.ToList();
        
        var email = claimList
            .FirstOrDefault(c => c.Type.Equals(
                "emails", 
                StringComparison.CurrentCultureIgnoreCase))
            ?.Value;

        if (email == null)
            return null;

        if (createUserIfNotExist)
            return await CreateUserIfNotExist(claimList); 
        
        var user = await Context.Users
            .FirstOrDefaultAsync(u => u.Email == email);

        return user;
    }
\end{lstlisting}


\subsection{Uploads mit Transaktionen}

Wenn man für einen Upload-Prozess von mehreren Bildern eine Transaktion möchte, wird eine 
``Transaktion'' für den Upload von Dateien erstellt, sodass, falls der Upload unterbrochen 
wird, der Prozess später fortgesetzt werden kann, ohne dass bereits hochgeladene Dateien 
erneut hochgeladen werden müssen. Dafür wird ein Transaktionsobjekt in der Datenbank 
erstellt. (Siehe Kapitel \ref{chapter:datamodel}) Au\ss{}erdem kann es pro Benutzer:in nur 
eine offene Transaktion geben.

Wenn ein:e Benutzer:in eine neue Upload-Transaktion startet, wird eine Transaktion in der 
Datenbank erstellt. Diese Transaktion ist mit einem Fotoalbum des/r Benutzers:in und dem/r 
Benutzer:in selbst verknüpft und stellt sicher, dass alle Fotos, die hochgeladen werden, 
im Kontext dieses Albums gespeichert werden.

Als nächsten können eine Reihe an Bildern nacheinnander hochgeladen werden. 
Falls der Upload unterbrochen wird, weil beispielsweise die Verbindung abbricht oder der:die 
Benutzer:in die Sitzung beendet, bleibt die Transaktion vorhanden. Später wird dann im Frontend
überprüft, welche Fotos schon hochgeladen wurden und die Bilder, welche noch nicht hochgeladen 
wurden, können sie in einem weiteren Upload-Versuch erneut hochgeladen werden.

Zum Schluss wird dann die Transaktion in der Datenbank gelöscht, wodurch dann klar ist, dass
die Transaktion abgeschlossen ist.

\subsubsection{Implementierung der Transaktions-Verwaltungs-Endpoints}

Dieser Endpoint (Sieh Listing \ref{lst:method-create-transaction}) ist für das Eröffnen einer neuen Transaktion zuständig. Es wird zunächst 
geprüft, ob der:die Benutzer:in authentifiziert ist und wenn dies der Fall ist, wird überprüft, 
ob der:die Benutzer:in bereits eine offene Transaktion hat. Falls ja, wird eine Fehlermeldung 
zurückgegeben. Danach wird überprüft, ob das angegebene Fotoalbum existiert. Wenn das 
Album nicht gefunden wird, gibt es eine BadRequest-Antwort. 

Nachdem alle Validierungen erfolgreich sind, wird eine neue Transaktion in der 
Datenbank erstellt und gespeichert, die mit dem Fotoalbum des/r Benutzers:in verknüpft ist.
Schlussendlich, wenn die Transaktion erfolgreich erstellt wurde, erhält der:die Benutzer:in 
eine Created-Antwort.

\begin{lstlisting}[numbers=left,caption={Methode OpenTransaction},label={lst:method-create-transaction}]
    [HttpPost]
    [Authorize]
    [Route("transaction")]
    [RequiredScope("backend.write")]
    public async Task<Results<Created, BadRequest<string>, UnauthorizedHttpResult>> OpenTransaction([FromBody] PostTransactionDto postTransactionDto)
    {
        // check if the user is authenticated without errors
        var user = await UserSvc.CheckIfUserAuthenticated(User.Claims, true);
        
        // check if the user exists
        if (user == null) 
            // if user was not able created then the claims had an issue meaning unauthorized
            return TypedResults.Unauthorized();
        
        // check if a transaction is already open
        var dbTransaction = Context.Transactions
            .FirstOrDefault(t => t.UserId == user.Id);
        
        if (dbTransaction is not null)
            return TypedResults.BadRequest("A transaction already exists.");
        
        // check if the photoAlbum exists
        var album = Context.PhotoAlbums
            .FirstOrDefault(p => 
                p.Id == postTransactionDto.DestAlbumId && 
                p.UserId == user.Id);
        
        if (album is null)
            return TypedResults.BadRequest("PhotoAlbum does not exist");

        var transaction = new Transaction
        {
            PhotoAlbumId = postTransactionDto.DestAlbumId,
            UserId = user.Id
        };

        await Context.Transactions.AddAsync(transaction);
        await Context.SaveChangesAsync();
        
        return TypedResults.Created();
    }
\end{lstlisting}

Dieser Endpoint (Sieh Listing \ref{lst:method-delete-transaction}) wird verwendet, um eine Transaktion zu löschen. Zunächst gibt es wieder eine
Reihe an Validierungen. Es wird überprüft, ob der:die Benutzer:in authentifiziert ist. 
Dann wird überprüft, ob eine Transaktion mit der angegebenen Id existiert und ob der:die 
Benutzer:in der:die Besitzer:in dieser Transaktion ist. Falls die Transaktion gefunden 
wird, wird sie aus der Datenbank entfernt und die Änderungen werden gespeichert.

\begin{lstlisting}[numbers=left,caption={Methode DeleteTransactionById},label={lst:method-delete-transaction}]
    [HttpDelete]
    [Route("transaction/{transactionId:long}")]
    [Authorize]
    [RequiredScope("backend.write")]
    public async Task<Results<Ok, UnauthorizedHttpResult>> DeleteTransactionById(long transactionId)
    {
        // check if the user is authenticated without errors
        var user = await UserSvc.CheckIfUserAuthenticated(User.Claims);
        
        // check if the user exists
        if (user == null)
            return TypedResults.Unauthorized();
        
        // check if there are any transactions at all, for performance
        if (!Context.Transactions.Any()) 
            return TypedResults.Ok();
        
        // check if the transaction exists and if the user is the owner
        var transaction = Context.Transactions
            .FirstOrDefault(t => t.Id == transactionId && t.UserId == user.Id);
        
        if (transaction == null)
            return TypedResults.Ok();
        
        Context.Transactions.Remove(transaction);
        await Context.SaveChangesAsync();
            
        return TypedResults.Ok();
    }

\end{lstlisting}


\section{Security}
\label{sec:security}

Memoryland verlässt sich auf Azure AD B2C zur Authentifizierung und einer 
JWT-basierten Autorisierung für den API-Zugriff, um die Sicherheit der Daten
der Benutzer:innen zu gewährleisten.

\subsection{Authentifizierung}

Die Authentifizierung wird über Microsoft Identity Web und das JWT Bearer Token-Schema realisiert. 
Dabei wird die Token-Validierung so konfiguriert, dass Azure AD B2C als Identitätsanbieter 
genutzt wird. (siehe Listing \ref{lst:auth-init})

\begin{lstlisting}[numbers=left,caption={Authentifizierung-Initialisierung},label={lst:auth-init}]
builder.Services.AddAuthentication(JwtBearerDefaults.AuthenticationScheme)
    .AddMicrosoftIdentityWebApi(
        options => 
        {
            builder.Configuration.Bind("AzureAdB2C", options);
        },
        options =>
        {
            builder.Configuration.Bind("AzureAdB2C", options);
        });

\end{lstlisting}

\subsection{Autorisierung}

Die Autorisierung erfolgt über die Middleware ``UseAuthorization()'', die sicherstellt, 
dass nur authentifizierte Benutzer:innen Zugriff auf geschützte API-Endpunkte haben. 
(siehe Listing \ref{lst:auth-middleware-init})

\begin{lstlisting}[numbers=left,caption={Autorisierungs Middleware Initialisierung},label={lst:auth-middleware-init}]
app.UseAuthentication();
app.UseAuthorization();    
\end{lstlisting}

Zusätzlich kann die Autorisierung dann für bestimmte Endpunkte über 
das ``[Authorize]''-Attribut aktiviert werden. 
(siehe Listing \ref{lst:authorite-attribute-example})

\begin{lstlisting}[numbers=left,caption={Authorize-Attribut Beispiel},label={lst:authorite-attribute-example}]
[Authorize]
[HttpGet("secure-data")]
public IActionResult GetSecureData()
{
    return Ok("This is protected data.");
}    
\end{lstlisting}

