\setauthor{Isabel Schnalzenberger}

\section{Technologien}

\subsection{Angular}

\subsection{WebStorm}

\section{Frontend-Code Beschreibung}
\subsection{Redux-Architektur und seine drei Prinzipien}

% Links und info unter: https://2324-4bhif-wmc.github.io/2324-4bhif-wmc-lecture-notes/#_mvvm

\subsection{WebAPI Client Service}
% explain web api (explain toastservice?)

\subsection{Image Previews}
% explain image cropper -> preview image

\subsection{All Worlds Component}
% explain memoryland all worlds component
% create memoryland modal

\subsection{Edit Memoryland Config Component}
% explain edit memoryland config component

\subsection{Explore Worlds Component}
% explain memoryland explore worlds component

\subsection{Memoryland List Component}

% explain memoryland list component

\subsection{Integration der Unity WebGL}
\label{sec:frontend-integration-webgl}

% integration des code im Verzeichnis ./unity
% integration als iframe in den code der entsprechenden seite






\section{Benutzeroberfläche}

\subsection{Header}

\begin{figure} [h t]
    \centering
    \includegraphics[scale=0.5]{pics/header_login.PNG}
    \caption{Header des Frontend}
    \label{fig:header-frontend}
\end{figure}

Der Header einer mit Angular entwickelten Webanwendung enthält das Logo sowie 
den Namen der Website. Durch Klicken auf das Logo oder den Namen wird der 
Benutzer zur Hauptseite der Anwendung weitergeleitet.

Zusätzlich sind Navigationslinks integriert, die den Benutzer zu verschiedenen 
Bereichen der Website führen. Die verfügbaren Links umfassen ``Home'', ``Explore 
Worlds'', ``All Worlds'', ``Memory Store'' und ``About'', diese werden später noch 
genauer erklärt. Durch das Anklicken eines dieser Links wird die entsprechende 
Seite aufgerufen. Die Seiten ``Explore Worlds'', ``All Worlds'' und ``Memory Store'' 
jedoch sind nur dann zugänglich, wenn der Benutzer sich auf der Website 
eingeloggt.


\begin{figure} [h t]
    \centering
    \includegraphics[scale=0.4]{pics/benutzername_aendern.PNG}
    \caption{Benutzername Ändern}
    \label{fig:benutzername-aendern}
\end{figure}

Ein Login-Button ist ebenfalls Bestandteil des Headers. Falls ein Benutzer eingeloggt 
ist, wird vor dem Login-Button ein zusätzlicher Button angezeigt, der den aktuellen 
Benutzernamen als Label enthält. Ein Klick auf diesen Button führt zu einer separaten 
Seite, auf der der Benutzername geändert werden kann.

Die Änderung des Benutzernamens erfolgt durch direktes Bearbeiten eines Textfeldes. 
Um die Änderung zu übernehmen oder zur vorherigen Seite zurückzukehren, steht ein 
``Continue''-Button zur Verfügung. Dieser speichert die Anpassung und leitet den Benutzer 
wieder auf die Website weiter. Um den vorherigen Benutzernamen jedoch beizubehalten 
oder zur vorherigen Seite zurückzukehren, steht ein Cancel-Button zur Verfügung. 

Der Header bleibt auf allen, mit Angular entwickelten, Seiten der Anwendung unverändert. 
Diese Konsistenz gewährleistet eine einheitliche Benutzerführung und erleichtert die 
Navigation innerhalb der Anwendung.


\subsection{Home}

Das Layout der Startseite ist zentriert ausgerichtet. Im oberen Bereich der Seite befindet 
sich das Logo in gro\ss{}er Darstellung, ergänzt durch den Namen der Diplomarbeit. Direkt 
darunter wird eine Willkommensnachricht angezeigt (siehe Abbildung \ref{fig:homepage}).

\begin{figure} [h t]
    \centering
    \includegraphics[scale=0.4]{pics/home_page.PNG}
    \caption{Homepage}
    \label{fig:homepage}
\end{figure}

Am unteren Rand der Seite ist ein Link platziert, der zur Impressums- und 
Nutzungsbedingungen-Seite führt. Diese Seite ist unter der Sektion ``About'' 
näher erläutert. Alternativ kann dieselbe Seite über den About-Link im Header 
aufgerufen werden.

\subsection{About}

Die About-Seite umfasst das Impressum sowie die Nutzungsbedingungen und ist zentriert 
ausgerichtet.


Das Impressum enthält die Namen der Entwickler und deren Zuständigkeitsbereiche 
innerhalb der Website, insbesondere die Unterscheidung zwischen Frontend- und 
Backend-Entwicklung. Zusätzlich sind die vollständige Adresse der Schule mit 
Stra\ss{}enname, Hausnummer, Postleitzahl, Stadt und Land sowie die offiziellen 
Kontaktdaten der Schule, einschlie\ss{}lich Telefonnummer und E-Mail-Adresse, 
aufgeführt.

\begin{figure} [h t]
    \centering
    \includegraphics[scale=0.4]{pics/About_page_Impressum.PNG}
    \caption{About - Impressum}
    \label{fig:about-page-impressum}
\end{figure}



\begin{figure} [h t]
    \centering
    \includegraphics[scale=0.8]{pics/About_page_Nutzungsbedingungen.PNG}
    \caption{About - Nutzungsbedingungen}
    \label{fig:about-page-nutzungsbedingungen}
\end{figure}


Die Nutzungsbedingungen beinhalten allgemeine Vorschriften und Vereinbarungen zur 
Nutzung der Website. Sie umfassen spezifische Regelungen zu erlaubten und nicht 
erlaubten Inhalten, die technische Verfügbarkeit der Plattform sowie Bestimmungen 
zu unerwünschten Aktivitäten wie Spam, Flooding und Leeching. Zudem gibt es eine 
Sektion zur Verarbeitung und Nutzung von Nutzerdaten, die den Umgang mit 
personenbezogenen Informationen beschreibt.

Die Betreiber einer Website, wie der hier beschriebenen, müssen sich gegen den 
missbräuchlichen Umgang und die Verbreitung von verbotenen Materialien schützen. 
Derartige Nutzungsbedingungen gehören also genauso zum Umfang eines derartigen Projektes
wie die Implementierung selbst.






\subsection{Memory Store}

Der Memory Store ist in drei Abschnitte unterteilt und bietet verschiedene Funktionen 
zur Verwaltung von Fotoalben und Bildern. Die Benutzeroberfläche ist so gestaltet, 
dass alle wesentlichen Aktionen über interaktive Elemente, wie Buttons und 
Suchleisten, zugänglich sind.

\subsubsection{Meine Fotoalben - Das Menü}

\begin{wrapfigure}{r}{0.4\textwidth}
    \centering
    \includegraphics[scale=0.8]{pics/memory_store_menu.PNG}
    \caption{Memory Store Menü}
    \label{fig:memory-store-menu}
\end{wrapfigure}


Dieser Abschnitt enthält mehrere Buttons zur Verwaltung und Organisation von Fotoalben 
und hochzuladenen Bildern (siehe \ref{fig:memory-store-menu}).

Der erste Button mit der Bezeichnung ``Album Erstellen'' öffnet ein Pop-up-Fenster, 
in dem der Benutzer den Namen des neuen Albums in ein Textfeld eingeben kann. 
Falls der Erstellungsprozess abgebrochen werden soll, kann entweder der 
``Cancel''-Button oder das Schlie\ss{}symbol (X) oben rechts im Pop-up-Fenster gedrückt 
werden. Nach der Eingabe des gewünschten Namens wird das Album durch Drücken 
des ``Album Erstellen''-Buttons endgültig erstellt und der Liste der vorhandenen 
Alben hinzugefügt.


\emph{Foto Hochladen}

Der zweite Button mit der Bezeichnung ``Foto Hochladen'' ermöglicht das Hochladen 
einzelner Bilder in zuvor erstellte Alben. Nach Betätigung dieses Buttons öffnet 
sich ein Pop-up-Fenster, in dem zunächst ein Album aus einer Liste bestehender Alben 
ausgewählt werden muss. Anschlie\ss{}end kann eine Datei vom eigenen Computer durch 
Drücken des ``Browse''-Buttons ausgewählt werden. Sobald eine Datei ausgewählt wurde, 
wird ihr Name in einem Textfeld angezeigt. Der Benutzer hat an dieser Stelle die 
Möglichkeit, den Namen der Datei manuell zu ändern. Das Hochladen des ausgewählten 
Bildes erfolgt schlie\ss{}lich durch Drücken des ``Foto Hochladen''-Buttons am unteren 
Rand des Pop-ups (siehe \ref{fig:memory-store-foto-hochladen}).

\begin{wrapfigure}{r}{0.4\textwidth}
    \centering
    \includegraphics[scale=0.4]{pics/memory_store_teil1_button2.PNG}
    \caption{Memory Store - Foto Hochladen}
    \label{fig:memory-store-foto-hochladen}
\end{wrapfigure}


\emph{Album Hochladen}


Der dritte Menüpunkt mit der Bezeichnung ``Album Hochladen'' ermöglicht das 
Hochladen eines gesamten Albums bzw. eines lokalen Ordners mit mehreren Bildern. 
Beim Anklicken öffnet sich ein Pop-up-Fenster, in dem zunächst ein Zielalbum aus 
der Liste zuvor erstellter Alben ausgewählt werden muss. Anschlie\ss{}end kann ein 
Album vom lokalen Computer über den ``Browse''-Button ausgewählt werden. Nach der 
Auswahl wird angezeigt, wie viele Bilder sich in dem gewählten Album befinden. 
Zusätzlich gibt es eine Checkbox, die – falls aktiviert – einen resumable Upload 
ermöglicht. Diese Funktion erlaubt es, den Upload-Prozess bei einer Unterbrechung 
fortzusetzen.

\begin{wrapfigure}{r}{0.4\textwidth}
    \centering
    \includegraphics[scale=0.4]{pics/memory_store_teil1_button3.PNG}
    \caption{Memory Store - Album Hochladen}
    \label{fig:memory-store-album-hochladen}
\end{wrapfigure}

Nachdem der Benutzer den ``Fotos Hochladen''-Button betätigt, wird das aktuelle 
Pop-up-Fenster durch ein neues ersetzt, das eine Fortschrittsanzeige (Progress Bar) 
enthält. Diese zeigt an, wie viele der Bilder bereits hochgeladen wurden. Sobald 
der Upload abgeschlossen ist, kann der Vorgang durch Drücken des ``Exit''-Buttons 
beendet werden.

Falls der Upload während des Prozesses unterbrochen wird, kann er später fortgesetzt 
werden. Eine Unterbrechung erfolgt entweder durch das Drücken des ``Exit''-Buttons 
oder durch das Klicken au\ss{}erhalb des Pop-ups. Voraussetzung für die Fortsetzung ist, 
dass die resumable Upload-Funktion zuvor aktiviert wurde. In diesem Fall kann der 
Upload über den vierten Button ``Resume Upload'' erneut gestartet werden. Um den 
Upload fortzusetzen, muss das zuvor ausgewählte Album, von dem die Bilder stammen 
(footnote siehe discord), erneut ausgewählt werden. Nach der Auswahl kann man den 
Upload weiterführen. Während sich ein Upload im gestoppten Zustand befindet, ist das 
Hochladen weiterer Alben nicht möglich bis der laufende Upload abgeschlossen wurde.

\subsubsection{Alben}

Dieser Abschnitt enthält eine Suchleiste, mit der der Benutzer nach einem bestimmten 
Album suchen kann. Während der Eingabe von Zeichen oder Wörtern werden, automatisch 
alle Alben angezeigt, die den eingegebenen Begriff enthalten. Diese Echtzeit-Suche 
erleichtert die Navigation innerhalb der gespeicherten Alben.

\begin{wrapfigure}{r}{0.4\textwidth}
    \centering
    \includegraphics[scale=0.5]{pics/memory_store_teil2.PNG}
    \caption{Memory Store - Alben}
    \label{fig:memory-store-alben}
\end{wrapfigure}

Direkt unter der Suchleiste befindet sich eine Liste aller vom Benutzer erstellten 
Alben. Jedes Album wird mit seinem Namen angezeigt und hat zwei interaktive Symbole 
neben sich.

Das Stift-Symbol ermöglicht das Umbenennen des Albums. Durch Klicken auf das Symbol 
öffnet sich ein Pop-up-Fenster mit einem Textfeld, in das der neue Name eingegeben 
und bestätigt werden kann.

Das Papierkorb-Symbol ermöglicht das Löschen des Albums. Sobald der Button gedrückt 
wird, wird das Album ohne weitere Bestätigung gelöscht. Alle darin gespeicherten 
Fotos werden damit ebenfalls entfernt.

Falls ein Album durch Anklicken ausgewählt wird, erscheint sein Inhalt im dritten 
Abschnitt, der die enthaltenen Fotos anzeigt.

\subsubsection{Fotos}

\begin{figure} [h t]
    \centering
    \includegraphics[scale=0.7]{pics/memory_store_teil3.PNG}
    \caption{Memory Store - Fotos}
    \label{fig:memory-store-fotos}
\end{figure}

Im oberen Bereich befindet sich eine Suchleiste, die identisch zur Suchfunktion in 
Abschnitt 2 arbeitet. Hier kann der Benutzer gezielt nach bestimmten Fotos innerhalb 
des gewählten Albums suchen.

Wenn kein Album in Abschnitt ``Alben'' ausgewählt wurde, bleibt dieser Bereich leer. 
Sobald jedoch ein Album ausgewählt wurde, werden die darin enthaltenen Fotos im 
unteren Bereich inklusive kleiner Vorschaubilder aufgelistet.

Jedes Foto wird mit seinem Namen angezeigt und besitzt zwei interaktive Symbole direkt 
daneben.

Das Stift-Symbol ermöglicht das Umbenennen des Bildes. Durch Anklicken wird ein Textfeld 
geöffnet, in das der neue Name eingetragen werden kann.

Das Papierkorb-Symbol löscht das Foto dauerhaft. Das Bild wird nach Betätigung des Symbols 
ohne weitere Bestätigung aus dem Album entfernt.

Falls ein Foto durch Anklicken ausgewählt wird, öffnet sich eine Ansicht mit dem Bilde, 
in der der Dateiname zusätzlich angezeigt wird. 

\subsection{All Worlds}

\begin{wrapfigure}{r}{0.4\textwidth}
    \centering
    \includegraphics[scale=0.7]{pics/all_worlds_teil1.PNG}
    \caption{All Worlds - Menü}
    \label{fig:all-worlds-menu}

    \vspace{1cm}

    \centering
    \includegraphics[scale=0.5]{pics/all_worlds_teil1_button.PNG}
    \caption{Memoryland erstellen}
    \label{fig:all-worlds-memoryland-erstellen}

\end{wrapfigure}

Die Erstellung und Verwaltung von Memorylands erfolgt ausschlie\ss{}lich auf der Seite 
``All Worlds''. Diese Seite ist in zwei Hauptabschnitte unterteilt, die 
unterschiedliche Funktionen bieten.

\subsubsection{Meine Erinnerungen}

% figure ist oben dabei
%\begin{wrapfigure}{r}{0.4\textwidth}
%\end{wrapfigure}

Der erste Abschnitt trägt die Bezeichnung ``Meine Erinnerungen''. Hier befindet 
sich ein Button mit der Aufschrift ``Memoryland Erstellen''. Durch das Anklicken 
dieses Buttons wird ein Pop-up-Fenster geöffnet, in dem verschiedene Einstellungen 
für das neue Memoryland vorgenommen werden können. Es enthält eine Auswahlmöglichkeit, 
in der der gewünschte Typ des Memorylands bestimmt werden kann. Zur Verfügung stehen 
zwei verschiedene Typen: ``Forest'' und ``Island''. Diese Typen definieren die 
visuelle Darstellung der Bilder innerhalb des Memorylands. Zusätzlich ist ein 
Textfeld vorhanden, in das der gewünschte Name des Memorylands eingetragen werden 
kann.



\subsubsection{Memorylands}

Der zweite Abschnitt der Seite trägt den Namen ``Memorylands'' und dient der 
Verwaltung bereits erstellter Memorylands. Um ein bestimmtes Memoryland schnell 
zu finden, steht eine Suchleiste zur Verfügung. Durch das Eintippen eines Namens 
oder einzelner Zeichen werden bereits während der Eingabe passende Ergebnisse 
gefiltert und angezeigt. Direkt unterhalb dieser Suchleiste werden alle zuvor 
erstellten Memorylands in einer strukturierten Übersicht dargestellt. Jedes 
Memoryland wird dabei mit mehreren Informationen versehen. Dazu gehören der 
Name des Memorylands, der gewählte Typ sowie die maximale Anzahl an Fotos, die 
innerhalb dieses Memorylands gespeichert werden können. Zusätzlich befinden sich 
neben diesen Informationen drei verschiedene Symbole, die für die Bearbeitung und 
Verwaltung der Memorylands vorgesehen sind.

\begin{figure} [h t]
    \centering
    \includegraphics[scale=0.6]{pics/all_worlds_teil2.PNG}
    \caption{Memorylands Übersicht}
    \label{fig:all-worlds-memorylands}
\end{figure}

Das mittlere Symbol ist ein Stift-Icon. Durch das Anklicken dieses Symbols wird 
die Möglichkeit geboten, den Namen des Memorylands zu ändern. Ein Eingabefeld 
erscheint, in das ein neuer Name eingetragen werden kann. Das nächste Symbol ist 
ein Papierkorb-Icon. Wenn dieses Symbol betätigt wird, wird das Memoryland 
unwiderruflich gelöscht. Eine zusätzliche Sicherheitsabfrage erfolgt dabei 
nicht, sodass das Memoryland sofort aus der Liste entfernt wird. 

Das forderste Symbol ist ein Schraubenschlüssel-Icon. Durch das Anklicken dieses 
Symbols öffnet sich ein weiteres Pop-up-Fenster, das zusätzliche Bearbeitungsoptionen 
bereitstellt. 

\subsubsection{Memorylands editieren}

In diesem Pop-up-Fenster kann ein bestehendes Fotoalbum aus der eigenen Sammlung 
ausgewählt werden, aus dem Bilder in das Memoryland eingefügt werden sollen. Um 
Bilder in das Memoryland zu integrieren, steht eine Drag-and-Drop-Funktion zur 
Verfügung. 

Bilder können aus einem oberen Bereich des Fensters in einen unteren 
Bereich verschoben werden. Falls sich eine gro\ss{}e Anzahl an Bildern in dem 
ausgewählten Album befindet, kann eine zusätzliche Suchleiste genutzt werden, 
um gezielt nach bestimmten Bildern zu suchen.

\begin{figure} [h t]
    \centering
    \includegraphics[scale=0.6]{pics/all_worlds_teil2_button.PNG}
    \caption{Memorylands editieren}
    \label{fig:all-worlds-memorylands-editieren}
\end{figure}

\subsection{Explore Worlds}


\begin{figure} [h t]
    \centering
    \includegraphics[scale=0.45]{pics/explore_worlds_header.PNG}
    \caption{Explore Worlds - Zugang}
    \label{fig:explore-worlds-overview}
\end{figure}

Nachdem ein Memoryland erfolgreich erstellt wurde, kann es über die Übersicht in ``All Worlds'' aufgerufen werden. 
Sobald der Nutzer auf den Namen eines Memoryland klickt, erfolgt eine automatische Weiterleitung 
zur Seite ``Explore Worlds''. Dort wird das ausgewählte Memoryland sofort geladen 
und visuell dargestellt. Dies ermöglicht eine interaktive Betrachtung der gespeicherten 
Erinnerungen innerhalb der gewählten Umgebung.


\begin{figure} [h t]
    \centering
    \includegraphics[scale=0.5]{pics/explore_worlds_loading_unity.PNG}
    \caption{Explore Worlds Loading Unity}
    \label{fig:explore-worlds-loading-unity}

    \centering
    \includegraphics[scale=0.5]{pics/explore_worlds_loading.PNG}
    \caption{Explore Worlds Loading}
    \label{fig:explore-worlds-loading}
\end{figure}

\begin{figure} [h t]
    \centering
    \includegraphics[scale=0.5]{pics/explore_worlds_forest.PNG}
    \caption{Explore Worlds Forest}
    \label{fig:explore-worlds-forest}

    
    \centering
    \includegraphics[scale=0.5]{pics/explore_worlds_island.PNG}
    \caption{Explore Worlds Island}
    \label{fig:explore-worlds-island}
\end{figure}
















\subsection{Toasts}

Toasts sind eine \emph{visuelle Rückmeldung} für Benutzer und zeigen an, ob eine Aktion 
erfolgreich war, fehlgeschlagen ist oder allgemeine Informationen enthält.  

Die Toast-Nachrichten erscheinen im oberen rechten Bereich der Website, jedoch 
unterhalb des Headers. Sie bleiben für eine bestimmte Zeit sichtbar und verschwinden 
automatisch. Alternativ können sie durch das Klicken auf ein ``X''-Symbol manuell 
geschlossen werden.  

Es gibt drei verschiedene Farbvarianten, die jeweils eine spezifische Bedeutung 
haben. Ein grüner Toast signalisiert ``Success'' und zeigt an, dass eine Aktion 
erfolgreich abgeschlossen wurde, beispielsweise die Erstellung eines Albums oder 
eines Memorylands. Ein roter Toast steht für ``Error'' und bedeutet, dass während 
eines Prozesses ein Fehler aufgetreten ist, wodurch der Vorgang unterbrochen oder 
gestoppt wurde. Ein blauer Toast steht für ``Information'' und enthält eine neutrale 
Nachricht, die lediglich eine Mitteilung darstellt, ohne dass es sich um eine 
Erfolgsmeldung oder eine Fehlermeldung handelt.

Durch die Farbkennzeichnung wird die Bedeutung der Toast-Nachricht sofort erkennbar. 
Dies verbessert die Benutzerfreundlichkeit, da Informationen schnell erfasst werden 
können, ohne dass zusätzliche Interaktionen erforderlich sind.







