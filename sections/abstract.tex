\begin{spacing}{1}
    \chapter*{Abstract}
\end{spacing}
\begin{wrapfigure}{r}{0.3\textwidth}
    \begin{center}
      \includegraphics[width=0.2\textwidth]{pics/memoryland-logo.png}
    \end{center}
\end{wrapfigure}
Brief summary of our amazing work. In English.
This is the only time we have to include a picture within the text.
The picture should somehow represent your thesis.
This is untypical for scientific work but required by the powers that are.
\lipsum[6]
\newpage
\begin{spacing}{1}
    \chapter*{Zusammenfassung}
\end{spacing}
\begin{wrapfigure}{r}{0.3\textwidth}
    \begin{center}
      \includegraphics[width=0.2\textwidth]{pics/memoryland-logo.png}
    \end{center}
\end{wrapfigure}
Zusammenfassung unserer genialen Arbeit. Auf Deutsch.
Das ist das einzige Mal, dass eine Grafik in den Textfluss eingebunden wird.
Die gewählte Grafik soll irgendwie eure Arbeit repräsentieren.
Das ist ungewöhnlich für eine wissenschaftliche Arbeit aber eine Anforderung der Obrigkeit.
\emph{Bitte auf keinen Fall mit der Zusammenfassung verwechseln, die den Abschluss der Arbeit bildet!}
\lipsum[6]
Unsere Diplomarbeit soll es Leuten (vor allem Familien) einfacher machen, schöne Erinnerungen leichter zu behalten und in einer schönen Ansicht in Reichweite zu haben. Um dies zu ermöglichen, soll man sich ein Video herunterladen können, welches die Fotos in Animationen aufbewahrt. Diese kann der User dann auch zum Beispiel auf einem online Verzeichnis ablegen, um diese nicht zu verlieren.
\\Unser Endziel besteht darin, eine vollständig funktionsfähige Website zu entwickeln, die es Benutzern ermöglicht, persönliche Erinnerungen in einem ansprechenden Format zu präsentieren. Hierbei wird eine immersive Erfahrung angestrebt, die es den Nutzern beispielsweise ermöglicht, ihre Erinnerungen in einem simulierten Achterbahnfahrt-Szenario zu durchlaufen.
  