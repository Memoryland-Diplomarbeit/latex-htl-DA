In Memoryland wird eine Web-API eingesetzt, um die Verwaltung von Bildern, Alben
und anderen Daten im Backend zu steuern. Die API übernimmt dabei die Speicherung, 
Strukturierung und den Zugriff auf diese Daten, während das Frontend über die API 
mit dem Backend kommuniziert.

\subsection{Arbeiten mit C\#-Controllern}

Die API wurde mittels einer C\# WebAPI mit Controllern entwickelt. Diese auf dem .NET-Framework
basierende Technologie ermöglicht es uns RESTful-Endpunkte zu erstellen, über die das Frontend
mit dem Backend kommunizieren kann. Controller helfen uns dabei die unterschiedlichen Funktionen
des Backends in eigene Klassen zu unterteilen, was eine klarere Struktur ermöglicht.
\footnote{Alle Infos zu C\# WebAPIs mit Controllern stammen von: \cite{MicrosoftCorporationaa}}

Um Code-Verdoppelung zu vermeiden, wurde im Memoryland-Backend eine Basisklasse 
für API-Controller implementiert. Von der Klasse ``\emph{ApiControllerBase}'' \ref{lst:api-controller-base} 
erben alle anderen Controller, wodurch eine einheitliche Struktur sichergestellt und die Wartung 
erleichtert wird.

Die Base-Controller Klasse ist mit den zwei Attributen ``[ApiController]'' und 
``[Route(``api/[controller]'')]'' versehen. Mit dem ``[ApiController]''-Attribut 
erkennt das .Net-Framework die Klasse als API-Controller. Das 
``[Route(``api/[controller]'')]''-Attribut hingegen legt die Basisroute für alle Endpunkte 
der Controller fest, die von der ApiControllerBase erben. 
Der Platzhalter \emph{controller} wird durch den Namen des jeweiligen Controllers ersetzt. 
Beispielsweise erhält die UploadController-Klasse dadurch die Basisroute ``api/Upload''.

\begin{lstlisting}[numbers=left,caption={APIControllerBase.cs},label={lst:api-controller-base}]
using Microsoft.AspNetCore.Mvc;

namespace WebApi.Controllers;
    
[ApiController]
[Route("api/[controller]")]
public class ApiControllerBase : ControllerBase { }    
\end{lstlisting}

\subsubsection{Probleme bei der Implementierung}

Bei der Implementierung dieser Struktur kam es zu dem Problem, dass in ein paar Methodenrouten
ein führende Schrägstrich ``\slash'' verwendet wurde. Dies überschreibt jedoch die definierte
Basisroute des Controllers.

Beispielsweise führt die Angabe ``[HttpPost(``/picture'')]'' dazu, dass die Route nicht 
``api/Upload/picture'' lautet, sondern nur ``/picture''. Daher sollte in der Methodenroute 
kein führender Schrägstrich ``\slash'' verwendet werden.

Dies führte zu Problemen auch mit Swagger und OpenAPI, da diese Middleware versuchte, 
die Methode dem entsprechenden Controller zuzuordnen. Da der Controller eine Basisroute 
definiert hatte, jedoch die tatsächliche Route aufgrund des beschriebenen Problems nicht 
in diese Struktur eingebunden war, warf die Middleware den Error. Dies führte dazu, 
dass Swagger und OpenAPI die Methode nicht korrekt dokumentieren oder aufrufen konnten.

\subsection{Endpunkte in C\# Controllern}

Die Methode MyPostEndpointMethod \ref{lst:endpoint-example} stellt einen HTTP-POST-Endpunkt in einer Web-API bereit und
wird als Beispiel für die folgende Erklärung der Implementierung von Endpunkten in C\# Controllern
verwendet. Die Methode ist mit mehreren Attributen versehen, die Authentifizierung, 
Autorisierung und die API Route steuern. Das ``[Authorize]''-Attribut stellt sicher, 
dass nur authentifizierte Benutzer Zugriff auf die Methode erhalten, wobei es dafür
die in der Datei ``Program.cs'' definierte Technologie verwendet. Zusätzlich überprüft 
``[RequiredScope(``backend.write'')]'', ob der Benutzer über die notwendigen Berechtigungen 
verfügt. Diese Überprüfung basiert auf OAuth 2.0 und OpenID Connect und stellt sicher, 
dass nur Benutzer mit den nötigen Rechten (Scopes) bestimmte Operationen durchführen können. 

Mit dem ``[Route(...)]''-Attribut wird die Route der Methode definiert. Hierbei können in
geschwungenen Klammern ``\{ \}'' URL-Parameter definiert werden. Es können aber auch 
Objekte im Body des HTTP-Aufrufs mitgegeben werden. Dafür benötigt man das 
Attribut ``[FromBody]'' vor dem Parameter der Funktion setzen. Das JSON-Objekt wird dann
automatisch in die Objekt-Instanz deserialisiert und kann dann normal genutzt werden. 

Schlussendlich gibt die Methode dann ein Results-Objekt zurück, welches mehrere 
mögliche Antworttypen umfasst. Diese Antworttypen beinhalten die HTTP-Status-Werte.
Bei erfolgreicher Erstellung eines neuen Objekts wird also ein Created-Ergebnis zurückgegeben. 
Dies entspricht dem Status-Code 201. Falls die Anfrage fehlerhafte oder unvollständige Daten 
enthält, wird ein BadRequest<string> zurückgeliefert, welches auch zusätzliche Informationen
im ``string''-Format enthalten kann. Ist die Verarbeitung erfolgreich, aber kein neues Objekt 
erforderlich, wird stattdessen ein Ok<ReturnObj> zurückgegeben, welches wieder bestimmte
Typen enthalten kann, wie eine Liste an Bildern.
\footnote{Alle Informationen zur Implementierung von Endpunkten in C\# Controllern stammen von: \cite{MicrosoftCorporationaa} \cite{MicrosoftCorporationab}}

\begin{lstlisting}[numbers=left,caption={Beispiel eines Endpunkts},label={lst:endpoint-example}]
[HttpPost]
[Authorize]
[Route("mypath/{objId:long}")]
[RequiredScope("backend.write")]
public async Task<Results<Created, BadRequest<string>, Ok<ReturnObj>>> 
    MyPostEndpointMethod(
    long objId, 
    [FromBody] ObjDto postObjDto)
{
    ...
}
\end{lstlisting}

\subsection{API-Dokumentation mit Swagger und OpenApi}

Für eine übersichtlichere Dokumentation der API, wurde Swagger und OpenAPI eingesetzt.
OpenAPI ist eine Spezifikation zur Beschreibung und Dokumentationvon REST-APIs. Eine
OpenAPI-Datei enthält alle Informationen zu der API, einschließlich der verfügbaren 
Endpunkte, sowie der Ein- und Ausgaben. Diese JSON Spezifikation wird in diesem Projekt
automatisch erstellt.

Swagger ist eine Sammlung von Open-Source-Tools, die auf der OpenAPI Spezifikation 
aufbauen. Für diese Projekt wurde die Swagger UI verwendet, welches OpenAPI-Definitionen 
als interaktive Dokumentation darstellt, sodass API-Aufrufe direkt im Browser getestet 
werden können.
\footnote{Ein Beispiel einer Swagger-UI ist im Bild \ref{fig:swagger-ui-example} zu sehen.}

Durch den Einsatz von Swagger und OpenAPI wurde die Nutzung der API erleichtert, 
da die Entwickler die verfügbaren Endpunkte und deren Funktionalitäten direkt 
einsehen und testen konnten. 
\footnote{Alle Informationen zu Swagger und OpenAPI stammen von: \cite{SmartBearSoftware}}

\subsection{Erklärung der Controller und ihrer Aufgaben}

Die vier Controller in der API des Memoryland-Bakcends erfüllen jeweils spezifische 
Aufgaben im Zusammenhang mit der Verwaltung von Fotos, Alben, Uploads und deren Transaktionen.

\subsubsection{Album-Controller}

Der AlbumController ist zuständig für die Verwaltung von Fotoalben. Dieser Controller 
ermöglicht das Erstellen, Abrufen und Löschen von Alben, sowie das Zuweisen 
von Fotos zu Alben.

\begin{table}[h t]
    \centering
    \caption{Album Controller Endpunkte}
    \label{tab:album-controller}
    \begin{tabular}{|l|p{5cm}|l|p{5cm}|}
    \hline
    \textbf{Methode} & \textbf{Pfad} & \textbf{Authorized} & \textbf{Beschreibung} \\ \hline
    GET & /api/PhotoAlbum\break{/\{albumId\}} & Ja & Ruft ein Fotoalbum mit der angegebenen ID ab. \\ \hline
    GET & /api/PhotoAlbum & Ja & Ruft eine Liste aller Fotoalben ab. \\ \hline
    PUT & /api/PhotoAlbum & Ja & Bearbeitet den Namen eines Fotoalbums. \\ \hline
    POST & /api/PhotoAlbum\break{/\{albumName\}} & Ja & Erstellt ein neues Fotoalbum mit dem angegebenen Namen. \\ \hline
    DELETE & /api/PhotoAlbum\break{/\{photoAlbumId\}} & Ja & Löscht ein Fotoalbum mit der angegebenen ID. \\ \hline
    \end{tabular}
\end{table}


\subsubsection{Memoryland-Controller}

Der MemorylandController ist für die Verwaltung und Interaktion mit Memorylands 
verantwortlich. Dabei werden Memorylands, die Positionierung der Bilder in Memorylands
und Tokens für den Zugriff auf Memorylands verwaltet.

Konfigurationen von Memorylands sind hier beschrieben \ref{sec:memoryland-config}.

\begin{table}[h t]
    \centering
    \caption{Memoryland Controller Endpunkte}
    \label{tab:memoryland-controller}
    \begin{tabular}{|l|p{5cm}|l|p{5cm}|}
    \hline
    \textbf{Methode} & \textbf{Pfad} & \textbf{Authorized} & \textbf{Beschreibung} \\ \hline
    GET & /api/Memoryland/all & Ja & Gibt eine Liste aller Memorylands zurück. \\ \hline
    GET & /api/Memoryland & Nein & Gibt eine Memoryland mit all dessen Bildern zurück. Dafür wird ein Token benötigt \\ \hline
    PUT & /api/Memoryland & Ja & Bearbeitet den Namen eines Memorylands. \\ \hline
    GET & /api/Memoryland/\{id\}\break/configuration & Ja & Gibt die Konfiguration für das angegebene Memoryland zurück. \\ \hline
    GET & /api/Memoryland/types & Nein & Gibt die verfügbaren Typen für Memorylands zurück. \\ \hline
    GET & /api/Memoryland/\{id\}\break/token & Ja & Gibt das Token für das angegebene Memoryland zurück. \\ \hline
    POST & /api/Memoryland/\{id\}\break/token & Ja & Erstellt ein neues Token für das angegebene Memoryland. \\ \hline
    POST & /api/Memoryland\break/\{memorylandName\}\break/{memorylandTypeId} & Ja & Erstellt ein Memoryland mit dem angegebenen Namen und Typ. \\ \hline
    POST & /api/Memoryland\break/\{memorylandId\} & Ja & Erstellt eine Konfiguration für das Memoryland. \\ \hline
    DELETE & /api/Memoryland\break/\{memorylandId\} & Ja & Löscht das angegebene Memoryland. \\ \hline
    DELETE & /api/Memoryland/config\break/\{id\} & Ja & Löscht die angegebene Konfiguration eines Memorylands. \\ \hline
    \end{tabular}
\end{table}


\subsubsection{Foto-Controller}

Der FotoController ist für die Verwaltung von Fotos verantwortlich. Dies umfasst das 
Abrufen, Erstellen und Löschen von Fotos, sowie das Verwalten der Fotodaten, die in 
Alben gespeichert sind. \ref{tab:foto-controller}

\begin{table}[h t]
    \centering
    \caption{Foto Controller Endpunkte}
    \label{tab:foto-controller}
    \begin{tabular}{|l|p{5cm}|l|p{5cm}|}
    \hline
    \textbf{Methode} & \textbf{Pfad} & \textbf{Authorized} & \textbf{Beschreibung} \\ \hline
    GET & /api/Photo/\{albumId\}\break/\{photoName\} & Ja & Ruft ein Foto aus einem Album ab. \\ \hline
    DELETE & /api/Photo/\{photoId\} & Ja & Löscht ein Foto mit der angegebenen ID. \\ \hline
    PUT & /api/Photo & Ja & Bearbeitet den Namen eines Fotos. \\ \hline
    \end{tabular}
\end{table}


\subsubsection{Upload-Controller}

Der UploadController ist für das Hochladen von Fotos und das Öffnen von 
Transaktionen zuständig. \ref{tab:upload-controller}

\begin{table}[h t]
    \centering
    \caption{Upload Controller Endpunkte}
    \label{tab:upload-controller}
    \begin{tabular}{|l|p{5cm}|l|p{5cm}|}
    \hline
    \textbf{Methode} & \textbf{Pfad} & \textbf{Authorized} & \textbf{Beschreibung} \\ \hline
    GET & /api/Upload/transaction & Ja & Ruft alle Upload-Transaktionen ab. \\ \hline
    POST & /api/Upload/transaction & Ja & Erstellt eine neue Upload-Transaktion. \\ \hline
    POST & /api/Upload/photo & Ja & Lädt ein Foto hoch. \\ \hline
    DELETE & /api/Upload/transaction\break/\{transactionId\} & Ja & Löscht eine Upload-Transaktion mit der angegebenen ID. \\ \hline
    \end{tabular}
\end{table}


