\subsection{.NET mit C\#}

C\# ist eine objektorientierte Programmiersprache, die speziell für die Entwicklung von 
Anwendungen auf der .NET-Plattform optimiert wurde. Als eine der fünf beliebtesten 
Programmiersprachen auf GitHub ist sie vollständig ``open-source''.

C\# greif die Konzepte aus JavaScript, Java und C++ auf, was sie für Leute mit Kenntnissen
in diesen Bereichen leichter zu lernen macht. Zu den zentralen Sprachmerkmalen gehören Generics, 
Pattern Matching, asynchrone Programmierung und Records. Diese Funktionen ermöglichen eine 
typsichere und strukturierte Entwicklung.

Für die Arbeit mit C\# stehen verschiedene Entwicklungsumgebungen und Werkzeuge zur Verfügung. 
Visual Studio bietet eine integrierte Entwicklungsumgebung mit umfangreichen Funktionen. 
Visual Studio Code stellt eine leichtgewichtige Alternative mit Erweiterungsmöglichkeiten 
dar. Zudem ermöglichen Kommandozeilen-Tools (CLI-Tools) eine flexible Nutzung. JetBrains Rider 
ist eine weitere IDE, die speziell für .NET optimiert wurde und in diesem Projekt genutzt 
wird. \ref{subsection:jetbrains-rider}

In dieser Diplomarbeit wurde C\# für das REST-Backend verwendet. Die zentrale Datenverwaltung 
erfolgte über Entity Framework Core mit einer PostgreSQL-Datenbank. Die Authentifizierung 
wurde durch die Anbindung an Azure AD B2C mittels MSAL realisiert. REST-APIs wurden implementiert, 
um die Kommunikation mit Unity und dem Frontend zu ermöglichen. Zusätzlich wurde Azure Blob 
Storage zur Speicherung und Verwaltung von Bilddateien integriert.

Die Wahl von C\# basiert auf zwei Faktoren. Die Sprache bietet eine direkte Integration mit 
Microsoft-Diensten, einschließlich der genutzten Azure-Dienste. Zudem wird C\# am meisten im 
Programmier-Zweig der HTL-Leonding unterrichtet, sodass bereits Vorkenntnisse vorhanden 
waren, die in der Diplomarbeit genutzt werden konnten.
\footnote{Alle Informationen zu C\# stammen von: \cite{MicrosoftCorporationo}}

\subsection{Entity Framework Core}

Entity Framework Core ist eine open-source Version des Entity Frameworks. Es dient 
als objektrelationaler Mapper (O\slash RM) für .NET-Anwendungen. Entity Framework Core ermöglicht 
die Arbeit mit relationalen Datenbanken unter Verwendung von .NET-Objekten und reduziert 
dabei den Aufwand für manuelle Datenzugriffslogik.

Der Datenzugriff erfolgt über ein Modell, das sich aus Entitätsklassen und einem 
Kontextobjekt zusammensetzt, welches bei uns ``ApplicationDbContext'' heißt. Dieses Kontextobjekt 
stellt eine Verbindung zur Datenbank her und verwaltet Datenbankabfragen sowie das Speichern 
von Daten. Entity Framework Core unterstützt verschiedene Ansätze zur Modellerstellung, 
darunter das Generieren eines Modells aus einer bestehenden Datenbank oder das manuelle 
Erstellen eines Modells, das anschließend mithilfe von Entity Framework Migrations in 
eine Datenbank überführt werden kann. In dieser Diplomarbeit wurde das Modell erstellt und dann
in eine Datenbank überführt.

Die Abfrage von Daten erfolgt über Language Integrated Query (LINQ). Änderungen an den Daten, 
wie das Erstellen, Modifizieren oder Löschen von Entitäten, werden über den DBContext 
vorgenommen und persistiert. Dabei stehen verschiedene Methoden zur Verfügung, 
um Daten effizient zu verwalten.
\footnote{Alle Informationen zu LINQ stammen von: \cite{MicrosoftCorporationq}}

Bei der Nutzung von Entity Framework Core sind verschiedene Aspekte zu beachten. Ein 
grundlegendes Verständnis relationaler Datenbanken ist erforderlich, um komplexe 
Datenbankoperationen effizient zu gestalten. Die Performance kann durch gezielte 
Optimierungen, wie den gezielten Einsatz von Indizes oder die Vermeidung von nicht 
skalierbaren Abfragen, verbessert werden.
\footnote{Alle Informationen zu Entity Framework Core stammen von: \cite{MicrosoftCorporationp}}


\subsection{MSAL}

MSAL (Microsoft Authentication Library) ermöglicht Entwicklern das Abrufen von Sicherheitstoken 
von der Microsoft-Identity-Platform, um Benutzer zu authentifizieren und auf gesicherte Web-APIs 
zuzugreifen. Sie kann verwendet werden, um sicheren Zugriff auf Microsoft Graph, andere 
Microsoft-APIs, Drittanbieter-Web-APIs oder eigene Web-APIs bereitzustellen. MSAL unterstützt 
viele verschiedene Anwendungsarchitekturen und Plattformen, darunter .NET, JavaScript, Java, 
Python, Android und iOS.

Für diese Diplomarbeit wurde MSAL verwendet, um eine benutzerdefinierte Authentifizierungslösung 
für das Angular-Frontend zu implementieren, die eine sichere Anmeldung und den Zugriff auf 
Web-APIs ermöglicht. Besonders hilfreich war MSAL bei der Anbindung an Azure AD B2C, da es 
eine einfache Handhabung der Authentifizierung und Tokenverwaltung ermöglichte.

Die Wahl von MSAL basiert auf der umfassenden Unterstützung durch Microsoft und der 
Integration in Azure-Dienste. Zudem wird MSAL in vielen verschiedenen Plattformen und 
Architekturen unterstützt, was es zu einer flexiblen Lösung für verschiedene Anwendungsfälle 
macht.
\footnote{Alle Informationen zu MSAL stammen von: \cite{MicrosoftCorporationmsal}}


\subsection{REST}

\subsection{Postgres-DB}
\label{subsection:postgres_db}

\subsection{Cloud-Dienste mit Azure}

\subsubsection{Azure Blob Storage}
\label{subsection:azure_blob_storage}

\subsubsection{Azure AD B2C}
\label{subsection:azure_ad_b2c}

https://learn.microsoft.com/en-us/azure/active-directory-b2c/overview

https://learn.microsoft.com/en-us/azure/active-directory-b2c/

\subsubsection{Azure WebApp}
\label{subsection:azure_web_app}

\subsubsection{Azure Static WebApp}
\label{subsection:azure_static_web_app}

\subsection{JetBrains Rider}
\label{subsection:jetbrains-rider}

\subsection{CI/CD mit GitHub Actions}
